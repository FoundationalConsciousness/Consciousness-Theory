% Options for packages loaded elsewhere
\PassOptionsToPackage{unicode}{hyperref}
\PassOptionsToPackage{hyphens}{url}
%
\documentclass[
]{article}
\usepackage{amsmath,amssymb}
\usepackage{lmodern}
\usepackage{iftex}
\ifPDFTeX
  \usepackage[T1]{fontenc}
  \usepackage[utf8]{inputenc}
  \usepackage{textcomp} % provide euro and other symbols
\else % if luatex or xetex
  \usepackage{unicode-math}
  \defaultfontfeatures{Scale=MatchLowercase}
  \defaultfontfeatures[\rmfamily]{Ligatures=TeX,Scale=1}
\fi
% Use upquote if available, for straight quotes in verbatim environments
\IfFileExists{upquote.sty}{\usepackage{upquote}}{}
\IfFileExists{microtype.sty}{% use microtype if available
  \usepackage[]{microtype}
  \UseMicrotypeSet[protrusion]{basicmath} % disable protrusion for tt fonts
}{}
\makeatletter
\@ifundefined{KOMAClassName}{% if non-KOMA class
  \IfFileExists{parskip.sty}{%
    \usepackage{parskip}
  }{% else
    \setlength{\parindent}{0pt}
    \setlength{\parskip}{6pt plus 2pt minus 1pt}}
}{% if KOMA class
  \KOMAoptions{parskip=half}}
\makeatother
\usepackage{xcolor}
\setlength{\emergencystretch}{3em} % prevent overfull lines
\providecommand{\tightlist}{%
  \setlength{\itemsep}{0pt}\setlength{\parskip}{0pt}}
\setcounter{secnumdepth}{-\maxdimen} % remove section numbering
\ifLuaTeX
  \usepackage{selnolig}  % disable illegal ligatures
\fi
\IfFileExists{bookmark.sty}{\usepackage{bookmark}}{\usepackage{hyperref}}
\IfFileExists{xurl.sty}{\usepackage{xurl}}{} % add URL line breaks if available
\urlstyle{same} % disable monospaced font for URLs
\hypersetup{
  hidelinks,
  pdfcreator={LaTeX via pandoc}}

\author{}
\date{}

\begin{document}

\hypertarget{ux433ux43bux430ux432ux430-7.-ux440ux430ux437ux43bux438ux447ux438ux435-ux43aux430ux43a-ux43fux435ux440ux432ux43eux43eux441ux43dux43eux432ux430}{%
\section{Глава 7. Различие как
первооснова}\label{ux433ux43bux430ux432ux430-7.-ux440ux430ux437ux43bux438ux447ux438ux435-ux43aux430ux43a-ux43fux435ux440ux432ux43eux43eux441ux43dux43eux432ux430}}

\hypertarget{section}{%
\subsubsection{}\label{section}}

\hypertarget{ux43eux442-ux444ux43eux440ux43c-ux43a-ux43fux435ux440ux432ux43eux43eux441ux43dux43eux432ux435}{%
\subsection{7.1. От форм к
первооснове}\label{ux43eux442-ux444ux43eux440ux43c-ux43a-ux43fux435ux440ux432ux43eux43eux441ux43dux43eux432ux435}}

В предыдущей главе знание было определено как устойчивая форма,
возникающая из конфигураций различий.

Однако возникает естественный вопрос: что именно является
\textbf{предельно фундаментальным} --- формы знания или сами различия?

В данной теории утверждается, что именно \textbf{различие}, а не форма и
не объект, является первоосновой онтологической структуры мира.

Формы, объекты и процессы являются вторичными по отношению к различиям,
поскольку без различий невозможно выделить ни форму, ни устойчивость, ни
структуру.

\hypertarget{ux440ux430ux437ux43bux438ux447ux438ux435-ux43fux440ux435ux434ux448ux435ux441ux442ux432ux443ux435ux442-ux43eux431ux44aux435ux43aux442ux443}{%
\subsection{7.2. Различие предшествует
объекту}\label{ux440ux430ux437ux43bux438ux447ux438ux435-ux43fux440ux435ux434ux448ux435ux441ux442ux432ux443ux435ux442-ux43eux431ux44aux435ux43aux442ux443}}

Классические онтологии исходят из существования объектов, обладающих
свойствами.\\
Различия в таком подходе рассматриваются как вторичные --- как различия
между уже существующими вещами.

Предлагаемая онтология инвертирует эту логику: объекты возникают
\textbf{из различий}, а не наоборот.

Объект есть нечто, что сохраняет определённость относительно других форм
знания.\\
Но сама возможность «относительно» предполагает уже заданные различия.

Следовательно, различие логически и онтологически предшествует объекту.

\hypertarget{ux43cux438ux43dux438ux43cux430ux43bux44cux43dux43eux435-ux440ux430ux437ux43bux438ux447ux438ux435}{%
\subsection{7.3. Минимальное
различие}\label{ux43cux438ux43dux438ux43cux430ux43bux44cux43dux43eux435-ux440ux430ux437ux43bux438ux447ux438ux435}}

Минимальное различие представляет собой наименьшую форму определённости,
допускающую выделение «не-тождества».

Это различие:

\begin{itemize}
\item
  не является числовым;
\item
  не имеет направления;
\item
  не предполагает пространства.
\end{itemize}

Минимальное различие --- это просто факт различённости, без каких-либо
дополнительных характеристик.

Именно такие различия лежат в основе последующей формализации, где они
будут интерпретироваться как элементы пространств состояний.

\hypertarget{ux440ux430ux437ux43bux438ux447ux438ux435-ux438-ux43eux442ux43dux43eux448ux435ux43dux438ux435}{%
\subsection{7.4. Различие и
отношение}\label{ux440ux430ux437ux43bux438ux447ux438ux435-ux438-ux43eux442ux43dux43eux448ux435ux43dux438ux435}}

Когда различий становится больше одного, между ними возникают
\textbf{отношения}.

Отношение не является дополнительной сущностью. Это способ совместного
существования различий.

Таким образом:

\begin{itemize}
\item
  различие первично;
\item
  отношение вторично;
\item
  структура возникает как сеть отношений между различиями.
\end{itemize}

Этот переход от различий к отношениям является ключевым для понимания
пространства и времени.

\hypertarget{ux43eux43dux442ux43eux43bux43eux433ux438ux44f-ux431ux435ux437-ux432ux435ux449ux435ux439}{%
\subsection{7.5. Онтология без
вещей}\label{ux43eux43dux442ux43eux43bux43eux433ux438ux44f-ux431ux435ux437-ux432ux435ux449ux435ux439}}

Принятие различия в качестве первоосновы позволяет построить онтологию
без фундаментальных «вещей».

В такой онтологии:

\begin{itemize}
\item
  нет неизменных сущностей;
\item
  нет абсолютных объектов;
\item
  устойчивость является динамической характеристикой.
\end{itemize}

Мир предстает не как совокупность вещей, а как \textbf{структура
различий}, находящихся в постоянном становлении.

\hypertarget{ux440ux430ux437ux43bux438ux447ux438ux435-ux438-ux438ux437ux43cux435ux440ux438ux43cux43eux441ux442ux44c}{%
\subsection{7.6. Различие и
измеримость}\label{ux440ux430ux437ux43bux438ux447ux438ux435-ux438-ux438ux437ux43cux435ux440ux438ux43cux43eux441ux442ux44c}}

Хотя различие в своей первичной форме не является количественным, оно
допускает введение меры.

Когда различия сравнимы, становится возможным:

\begin{itemize}
\item
  оценивать степень различия;
\item
  вводить метрики;
\item
  анализировать близость и удалённость.
\end{itemize}

Это положение является принципиальным мостом к математической
формализации, где различия будут выражены через информационные меры.

\hypertarget{ux440ux430ux437ux43bux438ux447ux438ux435-ux438-ux43dux435ux43eux43fux440ux435ux434ux435ux43bux451ux43dux43dux43eux441ux442ux44c}{%
\subsection{7.7. Различие и
неопределённость}\label{ux440ux430ux437ux43bux438ux447ux438ux435-ux438-ux43dux435ux43eux43fux440ux435ux434ux435ux43bux451ux43dux43dux43eux441ux442ux44c}}

Каждое различие возникает на фоне неопределённости.

Неопределённость не устраняется полностью с появлением различий, а лишь
локально уменьшается.

Фон неопределённости сохраняется и делает возможным дальнейшее
возникновение новых различий.

Таким образом, различие и неопределённость не противопоставлены, а
образуют диалектическую пару.

\hypertarget{ux440ux430ux437ux43bux438ux447ux438ux435-ux438-ux434ux438ux43dux430ux43cux438ux43aux430}{%
\subsection{7.8. Различие и
динамика}\label{ux440ux430ux437ux43bux438ux447ux438ux435-ux438-ux434ux438ux43dux430ux43cux438ux43aux430}}

Различия не являются статичными.

Они могут:

\begin{itemize}
\item
  усиливаться;
\item
  ослабевать;
\item
  исчезать;
\item
  порождать новые различия.
\end{itemize}

Эта динамика различий является прообразом времени, которое будет
рассмотрено в последующих главах.

Важно подчеркнуть, что динамика различий предшествует введению времени
как параметра.

\hypertarget{ux440ux430ux437ux43bux438ux447ux438ux435-ux43aux430ux43a-ux443ux43dux438ux432ux435ux440ux441ux430ux43bux44cux43dux44bux439-ux43fux440ux438ux43dux446ux438ux43f}{%
\subsection{7.9. Различие как универсальный
принцип}\label{ux440ux430ux437ux43bux438ux447ux438ux435-ux43aux430ux43a-ux443ux43dux438ux432ux435ux440ux441ux430ux43bux44cux43dux44bux439-ux43fux440ux438ux43dux446ux438ux43f}}

Различие является универсальным принципом, применимым ко всем уровням
реальности:

\begin{itemize}
\item
  квантовым состояниям;
\item
  физическим полям;
\item
  биологическим системам;
\item
  когнитивным процессам.
\end{itemize}

На каждом уровне различие принимает свои формы, но его онтологический
статус остаётся неизменным.

\hypertarget{ux437ux43dux430ux447ux435ux43dux438ux435-ux433ux43bux430ux432ux44b}{%
\subsection{7.10. Значение
главы}\label{ux437ux43dux430ux447ux435ux43dux438ux435-ux433ux43bux430ux432ux44b}}

В этой главе было показано, что:

\begin{itemize}
\item
  различие является первоосновой онтологии;
\item
  формы знания, объекты и структуры вторичны;
\item
  различия допускают измерение и формализацию.
\end{itemize}

Тем самым подготовлен переход к следующему этапу онтологии ---
рассмотрению \textbf{пространства как структуры различий}.

\end{document}
