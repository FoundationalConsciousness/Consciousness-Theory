% Options for packages loaded elsewhere
\PassOptionsToPackage{unicode}{hyperref}
\PassOptionsToPackage{hyphens}{url}
%
\documentclass[
]{article}
\usepackage{amsmath,amssymb}
\usepackage{lmodern}
\usepackage{iftex}
\ifPDFTeX
  \usepackage[T1]{fontenc}
  \usepackage[utf8]{inputenc}
  \usepackage{textcomp} % provide euro and other symbols
\else % if luatex or xetex
  \usepackage{unicode-math}
  \defaultfontfeatures{Scale=MatchLowercase}
  \defaultfontfeatures[\rmfamily]{Ligatures=TeX,Scale=1}
\fi
% Use upquote if available, for straight quotes in verbatim environments
\IfFileExists{upquote.sty}{\usepackage{upquote}}{}
\IfFileExists{microtype.sty}{% use microtype if available
  \usepackage[]{microtype}
  \UseMicrotypeSet[protrusion]{basicmath} % disable protrusion for tt fonts
}{}
\makeatletter
\@ifundefined{KOMAClassName}{% if non-KOMA class
  \IfFileExists{parskip.sty}{%
    \usepackage{parskip}
  }{% else
    \setlength{\parindent}{0pt}
    \setlength{\parskip}{6pt plus 2pt minus 1pt}}
}{% if KOMA class
  \KOMAoptions{parskip=half}}
\makeatother
\usepackage{xcolor}
\setlength{\emergencystretch}{3em} % prevent overfull lines
\providecommand{\tightlist}{%
  \setlength{\itemsep}{0pt}\setlength{\parskip}{0pt}}
\setcounter{secnumdepth}{-\maxdimen} % remove section numbering
\ifLuaTeX
  \usepackage{selnolig}  % disable illegal ligatures
\fi
\IfFileExists{bookmark.sty}{\usepackage{bookmark}}{\usepackage{hyperref}}
\IfFileExists{xurl.sty}{\usepackage{xurl}}{} % add URL line breaks if available
\urlstyle{same} % disable monospaced font for URLs
\hypersetup{
  hidelinks,
  pdfcreator={LaTeX via pandoc}}

\author{}
\date{}

\begin{document}

\hypertarget{ux433ux43bux430ux432ux430-13.-ux441ux43cux44bux441ux43b-ux446ux435ux43dux43dux43eux441ux442ux44c-ux438-ux43dux430ux43fux440ux430ux432ux43bux435ux43dux438ux435-ux440ux430ux437ux432ux438ux442ux438ux44f}{%
\section{Глава 13. Смысл, ценность и направление
развития}\label{ux433ux43bux430ux432ux430-13.-ux441ux43cux44bux441ux43b-ux446ux435ux43dux43dux43eux441ux442ux44c-ux438-ux43dux430ux43fux440ux430ux432ux43bux435ux43dux438ux435-ux440ux430ux437ux432ux438ux442ux438ux44f}}

\hypertarget{ux43fux440ux43eux431ux43bux435ux43cux430-ux441ux43cux44bux441ux43bux430-ux432-ux444ux443ux43dux434ux430ux43cux435ux43dux442ux430ux43bux44cux43dux43eux439-ux43eux43dux442ux43eux43bux43eux433ux438ux438}{%
\subsection{13.1. Проблема смысла в фундаментальной
онтологии}\label{ux43fux440ux43eux431ux43bux435ux43cux430-ux441ux43cux44bux441ux43bux430-ux432-ux444ux443ux43dux434ux430ux43cux435ux43dux442ux430ux43bux44cux43dux43eux439-ux43eux43dux442ux43eux43bux43eux433ux438ux438}}

В традиционных картинах мира смысл либо вводится извне (религиозные и
телеологические системы), либо редуцируется к субъективному переживанию
или социальной конструкции.

В научных онтологиях смысл, как правило, исключается из описания
реальности, считаясь нефундаментальным понятием.

В Теории Сознания возникает возможность понять смысл как онтологическую
категорию,\\
не вводя внешних целей и предписаний.

\hypertarget{ux441ux43cux44bux441ux43b-ux43aux430ux43a-ux441ux442ux440ux443ux43aux442ux443ux440ux430-ux437ux43dux430ux43dux438ux44f}{%
\subsection{13.2. Смысл как структура
знания}\label{ux441ux43cux44bux441ux43b-ux43aux430ux43a-ux441ux442ux440ux443ux43aux442ux443ux440ux430-ux437ux43dux430ux43dux438ux44f}}

Смысл не является объектом и не существует сам по себе.

Смысл есть \textbf{способ организации различий} в структуре знания.

Там, где различия образуют устойчивые, взаимосогласованные конфигурации,
возникает смысл.

Смысл не добавляется к знанию, а является его внутренней
характеристикой.

\hypertarget{ux441ux43cux44bux441ux43b-ux438-ux440ux430ux437ux43bux438ux447ux438ux435}{%
\subsection{13.3. Смысл и
различие}\label{ux441ux43cux44bux441ux43b-ux438-ux440ux430ux437ux43bux438ux447ux438ux435}}

Различия сами по себе ещё не образуют смысла.

Смысл возникает, когда различия:

\begin{itemize}
\item
  связаны,
\item
  соотнесены,
\item
  направлены друг относительно друга.
\end{itemize}

Таким образом, смысл есть \textbf{направленная структура различий}.

Он отличается от информации тем, что несёт внутреннюю связность, а не
просто разнообразие состояний.

\hypertarget{ux446ux435ux43dux43dux43eux441ux442ux44c-ux43aux430ux43a-ux443ux441ux442ux43eux439ux447ux438ux432ux43eux441ux442ux44c-ux441ux43cux44bux441ux43bux430}{%
\subsection{13.4. Ценность как устойчивость
смысла}\label{ux446ux435ux43dux43dux43eux441ux442ux44c-ux43aux430ux43a-ux443ux441ux442ux43eux439ux447ux438ux432ux43eux441ux442ux44c-ux441ux43cux44bux441ux43bux430}}

Ценность в Теории Сознания не является моральной категорией.

Ценность определяется как \textbf{степень устойчивости и
воспроизводимости смысла}\\
в динамике Сознания.

Те структуры знания, которые сохраняются, развиваются и интегрируются в
более широкие конфигурации, обладают большей ценностью.

Ценность не предписывается, а выявляется динамикой.

\hypertarget{ux43dux430ux43fux440ux430ux432ux43bux435ux43dux438ux435-ux440ux430ux437ux432ux438ux442ux438ux44f}{%
\subsection{13.5. Направление
развития}\label{ux43dux430ux43fux440ux430ux432ux43bux435ux43dux438ux435-ux440ux430ux437ux432ux438ux442ux438ux44f}}

Развитие в Теории Сознания не имеет заранее заданной цели.

Однако оно не является хаотичным. Направление развития определяется\\
тенденцией:

\begin{itemize}
\item
  к увеличению структурированности;
\item
  к усложнению различий;
\item
  к интеграции локальных смыслов в более общие структуры.
\end{itemize}

Развитие есть процесс обогащения пространства возможных смыслов.

\hypertarget{ux441ux432ux43eux431ux43eux434ux430-ux438-ux441ux43cux44bux441ux43b}{%
\subsection{13.6. Свобода и
смысл}\label{ux441ux432ux43eux431ux43eux434ux430-ux438-ux441ux43cux44bux441ux43b}}

Свобода, как было показано ранее, есть недоопределённость будущего.

Смысл возникает именно в этом пространстве: там, где возможны разные
пути, возникает различие значимости.

Если бы будущее было полностью предопределено, смысл утратил бы своё
онтологическое значение.

Таким образом, смысл и свобода являются взаимосвязанными аспектами одной
структуры реальности.

\hypertarget{ux43dux430ux431ux43bux44eux434ux430ux442ux435ux43bux44c-ux438-ux444ux43eux440ux43cux438ux440ux43eux432ux430ux43dux438ux435-ux441ux43cux44bux441ux43bux430}{%
\subsection{13.7. Наблюдатель и формирование
смысла}\label{ux43dux430ux431ux43bux44eux434ux430ux442ux435ux43bux44c-ux438-ux444ux43eux440ux43cux438ux440ux43eux432ux430ux43dux438ux435-ux441ux43cux44bux441ux43bux430}}

Наблюдатель, как локализация знания, не просто фиксирует реальность, а
участвует в формировании смыслов.

Акты внимания, выбора и интерпретации перестраивают структуру различий,
усиливая одни связи и ослабляя другие.

Смысл возникает на пересечении:

\begin{itemize}
\item
  структуры Сознания,
\item
  локализации знания,
\item
  динамики выбора.
\end{itemize}

\hypertarget{ux446ux435ux43dux43dux43eux441ux442ux438-ux438-ux43aux443ux43bux44cux442ux443ux440ux430}{%
\subsection{13.8. Ценности и
культура}\label{ux446ux435ux43dux43dux43eux441ux442ux438-ux438-ux43aux443ux43bux44cux442ux443ux440ux430}}

Коллективные структуры знания (язык, культура, наука) являются
носителями устойчивых смыслов.

Культура может быть понята как система стабилизации смыслов на
коллективном уровне.

Ценности общества отражают те формы знания, которые оказались наиболее
устойчивыми в исторической динамике.

\hypertarget{ux43aux43eux441ux43cux438ux447ux435ux441ux43aux43eux435-ux438ux437ux43cux435ux440ux435ux43dux438ux435-ux441ux43cux44bux441ux43bux430}{%
\subsection{13.9. Космическое измерение
смысла}\label{ux43aux43eux441ux43cux438ux447ux435ux441ux43aux43eux435-ux438ux437ux43cux435ux440ux435ux43dux438ux435-ux441ux43cux44bux441ux43bux430}}

Смысл не ограничивается человеческим опытом.

Вся Вселенная, как процесс саморазворачивания Сознания, обладает
внутренней направленностью к усложнению и структурированию.

Это не цель, а тенденция, возникающая из самой природы различий.

Смысл здесь --- не предписание, а форма внутренней согласованности
развития.

\hypertarget{ux432ux44bux432ux43eux434ux44b}{%
\subsection{13.10. Выводы}\label{ux432ux44bux432ux43eux434ux44b}}

В данной главе было показано, что:

\begin{itemize}
\item
  смысл является структурой знания;
\item
  ценность есть мера устойчивости смысла;
\item
  развитие имеет направленность без цели;
\item
  свобода создаёт пространство смыслов;
\item
  наблюдатель участвует в их формировании.
\end{itemize}

Смысл в Теории Сознания не навязывается реальности, а обнаруживается в
динамике различий.

\end{document}
