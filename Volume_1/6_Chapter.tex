% Options for packages loaded elsewhere
\PassOptionsToPackage{unicode}{hyperref}
\PassOptionsToPackage{hyphens}{url}
%
\documentclass[
]{article}
\usepackage{amsmath,amssymb}
\usepackage{lmodern}
\usepackage{iftex}
\ifPDFTeX
  \usepackage[T1]{fontenc}
  \usepackage[utf8]{inputenc}
  \usepackage{textcomp} % provide euro and other symbols
\else % if luatex or xetex
  \usepackage{unicode-math}
  \defaultfontfeatures{Scale=MatchLowercase}
  \defaultfontfeatures[\rmfamily]{Ligatures=TeX,Scale=1}
\fi
% Use upquote if available, for straight quotes in verbatim environments
\IfFileExists{upquote.sty}{\usepackage{upquote}}{}
\IfFileExists{microtype.sty}{% use microtype if available
  \usepackage[]{microtype}
  \UseMicrotypeSet[protrusion]{basicmath} % disable protrusion for tt fonts
}{}
\makeatletter
\@ifundefined{KOMAClassName}{% if non-KOMA class
  \IfFileExists{parskip.sty}{%
    \usepackage{parskip}
  }{% else
    \setlength{\parindent}{0pt}
    \setlength{\parskip}{6pt plus 2pt minus 1pt}}
}{% if KOMA class
  \KOMAoptions{parskip=half}}
\makeatother
\usepackage{xcolor}
\setlength{\emergencystretch}{3em} % prevent overfull lines
\providecommand{\tightlist}{%
  \setlength{\itemsep}{0pt}\setlength{\parskip}{0pt}}
\setcounter{secnumdepth}{-\maxdimen} % remove section numbering
\ifLuaTeX
  \usepackage{selnolig}  % disable illegal ligatures
\fi
\IfFileExists{bookmark.sty}{\usepackage{bookmark}}{\usepackage{hyperref}}
\IfFileExists{xurl.sty}{\usepackage{xurl}}{} % add URL line breaks if available
\urlstyle{same} % disable monospaced font for URLs
\hypersetup{
  hidelinks,
  pdfcreator={LaTeX via pandoc}}

\author{}
\date{}

\begin{document}

\hypertarget{ux433ux43bux430ux432ux430-6.-ux437ux43dux430ux43dux438ux435-ux43aux430ux43a-ux444ux43eux440ux43cux430}{%
\section{Глава 6. Знание как
форма}\label{ux433ux43bux430ux432ux430-6.-ux437ux43dux430ux43dux438ux435-ux43aux430ux43a-ux444ux43eux440ux43cux430}}

\hypertarget{section}{%
\subsubsection{}\label{section}}

\hypertarget{ux43eux442-ux440ux430ux437ux43bux438ux447ux438ux44f-ux43a-ux437ux43dux430ux43dux438ux44e}{%
\subsection{6.1. От различия к
знанию}\label{ux43eux442-ux440ux430ux437ux43bux438ux447ux438ux44f-ux43a-ux437ux43dux430ux43dux438ux44e}}

В предыдущей главе было показано, что самодифференциация Сознания
приводит к возникновению различий как первичных элементов
определённости.

Однако сами по себе различия ещё не образуют устойчивой структуры,
способной сохраняться, воспроизводиться и служить основанием для
дальнейшего развёртывания мира.

Для этого необходимо ввести понятие \textbf{знания}.

Знание в данной теории не является психологическим состоянием,
когнитивным процессом или результатом познания субъекта.

Оно представляет собой \textbf{онтологическую форму определённости},
возникающую внутри Сознания.

\hypertarget{ux43eux43dux442ux43eux43bux43eux433ux438ux447ux435ux441ux43aux43eux435-ux43eux43fux440ux435ux434ux435ux43bux435ux43dux438ux435-ux437ux43dux430ux43dux438ux44f}{%
\subsection{6.2. Онтологическое определение
знания}\label{ux43eux43dux442ux43eux43bux43eux433ux438ux447ux435ux441ux43aux43eux435-ux43eux43fux440ux435ux434ux435ux43bux435ux43dux438ux435-ux437ux43dux430ux43dux438ux44f}}

\textbf{Знание} --- это устойчивая конфигурация различий, сохраняющаяся
в процессе дальнейшей самодифференциации Сознания.

Ключевыми характеристиками знания являются:

\begin{itemize}
\item
  \textbf{устойчивость} --- способность сохранять структуру при
  изменениях;
\item
  \textbf{повторяемость} --- возможность воспроизведения;
\item
  \textbf{различимость} --- сохранение отличимости от других форм
  знания.
\end{itemize}

Знание не требует субъекта, интерпретатора или носителя.\\
Оно существует как форма внутри Сознания.

\hypertarget{ux43eux442ux43bux438ux447ux438ux435-ux437ux43dux430ux43dux438ux44f-ux43eux442-ux438ux43dux444ux43eux440ux43cux430ux446ux438ux438}{%
\subsection{6.3. Отличие знания от
информации}\label{ux43eux442ux43bux438ux447ux438ux435-ux437ux43dux430ux43dux438ux44f-ux43eux442-ux438ux43dux444ux43eux440ux43cux430ux446ux438ux438}}

Важно строго различать знание и информацию.

Информация всегда предполагает:

\begin{itemize}
\item
  носитель;
\item
  кодирование;
\item
  возможность передачи.
\end{itemize}

Знание в онтологическом смысле не требует ни носителя, ни передачи.\\
Информация является \textbf{производной формой знания}, возникающей
тогда, когда формы знания включаются в процессы кодирования и
взаимодействия.

Таким образом, информация не может быть первичной, поскольку она
предполагает уже существующие формы знания.

\hypertarget{ux444ux43eux440ux43cux430-ux43aux430ux43a-ux443ux441ux442ux43eux439ux447ux438ux432ux43eux441ux442ux44c-ux440ux430ux437ux43bux438ux447ux438ux439}{%
\subsection{6.4. Форма как устойчивость
различий}\label{ux444ux43eux440ux43cux430-ux43aux430ux43a-ux443ux441ux442ux43eux439ux447ux438ux432ux43eux441ux442ux44c-ux440ux430ux437ux43bux438ux447ux438ux439}}

Понятие формы вводится для описания устойчивости знания.

\textbf{Форма} --- это такая конфигурация различий, которая сохраняется
как целое при изменении своих компонентов.

Форма не тождественна геометрической фигуре или физическому объекту.\\
Это онтологическая характеристика устойчивости.

Контролируемая метафора: форма подобна мелодии, сохраняющейся при смене
отдельных нот, пока сохраняется их соотношение.

\hypertarget{ux432ux43eux437ux43dux438ux43aux43dux43eux432ux435ux43dux438ux435-ux441ux442ux440ux443ux43aux442ux443ux440}{%
\subsection{6.5. Возникновение
структур}\label{ux432ux43eux437ux43dux438ux43aux43dux43eux432ux435ux43dux438ux435-ux441ux442ux440ux443ux43aux442ux443ux440}}

Когда формы знания начинают соотноситься друг с другом, возникает
\textbf{структура}.

Структура представляет собой систему взаимосвязанных форм знания, в
которой:

\begin{itemize}
\item
  формы определяют друг друга;
\item
  различия приобретают контекст;
\item
  появляется возможность сложных конфигураций.
\end{itemize}

На этом этапе ещё нельзя говорить о пространстве и времени в привычном
смысле, однако уже возникает предпосылка для их появления.

\hypertarget{ux443ux441ux442ux43eux439ux447ux438ux432ux43eux441ux442ux44c-ux438-ux43eux43dux442ux43eux43bux43eux433ux438ux447ux435ux441ux43aux430ux44f-ux43fux430ux43cux44fux442ux44c}{%
\subsection{6.6. Устойчивость и онтологическая
память}\label{ux443ux441ux442ux43eux439ux447ux438ux432ux43eux441ux442ux44c-ux438-ux43eux43dux442ux43eux43bux43eux433ux438ux447ux435ux441ux43aux430ux44f-ux43fux430ux43cux44fux442ux44c}}

Устойчивость форм знания выполняет роль онтологической памяти.

Эта память:

\begin{itemize}
\item
  не локализована;
\item
  не является хранилищем;
\item
  не требует материального носителя.
\end{itemize}

Онтологическая память выражается в способности форм знания
воспроизводиться в процессе самодифференциации Сознания.

Именно эта память делает возможным существование законов,
закономерностей и повторяемых процессов.

\hypertarget{ux43eux442-ux444ux43eux440ux43cux44b-ux43a-ux43eux431ux44aux435ux43aux442ux443}{%
\subsection{6.7. От формы к
объекту}\label{ux43eux442-ux444ux43eux440ux43cux44b-ux43a-ux43eux431ux44aux435ux43aux442ux443}}

Объекты в привычном смысле возникают как частный случай форм знания
высокой устойчивости.

То, что в физике называется объектом, представляет собой форму знания,
устойчивую настолько, что она воспринимается как независимая и внешняя
по отношению к наблюдателю.

Таким образом, объектность является \textbf{производным онтологическим
режимом}, а не фундаментальной характеристикой реальности.

\hypertarget{ux437ux43dux430ux43dux438ux435-ux438-ux43dux430ux431ux43bux44eux434ux435ux43dux438ux435}{%
\subsection{6.8. Знание и
наблюдение}\label{ux437ux43dux430ux43dux438ux435-ux438-ux43dux430ux431ux43bux44eux434ux435ux43dux438ux435}}

Наблюдение представляет собой акт актуализации формы знания.

Наблюдатель не создаёт форму, а выбирает и фиксирует одну из возможных
форм знания, присутствующих в структуре Сознания.

Это положение имеет принципиальное значение для понимания квантовых
эффектов, где акт наблюдения влияет на проявление формы.

\hypertarget{ux43fux43eux434ux433ux43eux442ux43eux432ux43aux430-ux43a-ux444ux43eux440ux43cux430ux43bux438ux437ux430ux446ux438ux438}{%
\subsection{6.9. Подготовка к
формализации}\label{ux43fux43eux434ux433ux43eux442ux43eux432ux43aux430-ux43a-ux444ux43eux440ux43cux430ux43bux438ux437ux430ux446ux438ux438}}

Формы знания допускают:

\begin{itemize}
\item
  описание в терминах пространств состояний;
\item
  введение метрик различий;
\item
  анализ устойчивости и динамики.
\end{itemize}

В Томе II формы знания будут представлены как элементы математических
пространств, а их различия --- как меры информационного расхождения.

При этом важно подчеркнуть, что математическая формализация является
\textbf{описанием формы}, а не её причиной.

\hypertarget{ux437ux43dux430ux447ux435ux43dux438ux435-ux433ux43bux430ux432ux44b}{%
\subsection{6.10. Значение
главы}\label{ux437ux43dux430ux447ux435ux43dux438ux435-ux433ux43bux430ux432ux44b}}

В этой главе было показано, что:

\begin{itemize}
\item
  различия становятся знанием, когда приобретают устойчивость;
\item
  знание является онтологической формой, а не эпистемическим понятием;
\item
  формы знания лежат в основе объектов, законов и структур мира.
\end{itemize}

Тем самым создан фундамент для перехода к следующим ключевым понятиям:

\begin{itemize}
\item
  пространство как структура форм знания;
\item
  время как процесс изменения этих форм.
\end{itemize}

В следующей главе будет рассмотрено \textbf{различие как первооснова}, а
затем --- возникновение пространства и времени.

\end{document}
