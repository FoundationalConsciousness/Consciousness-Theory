% Options for packages loaded elsewhere
\PassOptionsToPackage{unicode}{hyperref}
\PassOptionsToPackage{hyphens}{url}
%
\documentclass[
]{article}
\usepackage{amsmath,amssymb}
\usepackage{lmodern}
\usepackage{iftex}
\ifPDFTeX
  \usepackage[T1]{fontenc}
  \usepackage[utf8]{inputenc}
  \usepackage{textcomp} % provide euro and other symbols
\else % if luatex or xetex
  \usepackage{unicode-math}
  \defaultfontfeatures{Scale=MatchLowercase}
  \defaultfontfeatures[\rmfamily]{Ligatures=TeX,Scale=1}
\fi
% Use upquote if available, for straight quotes in verbatim environments
\IfFileExists{upquote.sty}{\usepackage{upquote}}{}
\IfFileExists{microtype.sty}{% use microtype if available
  \usepackage[]{microtype}
  \UseMicrotypeSet[protrusion]{basicmath} % disable protrusion for tt fonts
}{}
\makeatletter
\@ifundefined{KOMAClassName}{% if non-KOMA class
  \IfFileExists{parskip.sty}{%
    \usepackage{parskip}
  }{% else
    \setlength{\parindent}{0pt}
    \setlength{\parskip}{6pt plus 2pt minus 1pt}}
}{% if KOMA class
  \KOMAoptions{parskip=half}}
\makeatother
\usepackage{xcolor}
\setlength{\emergencystretch}{3em} % prevent overfull lines
\providecommand{\tightlist}{%
  \setlength{\itemsep}{0pt}\setlength{\parskip}{0pt}}
\setcounter{secnumdepth}{-\maxdimen} % remove section numbering
\ifLuaTeX
  \usepackage{selnolig}  % disable illegal ligatures
\fi
\IfFileExists{bookmark.sty}{\usepackage{bookmark}}{\usepackage{hyperref}}
\IfFileExists{xurl.sty}{\usepackage{xurl}}{} % add URL line breaks if available
\urlstyle{same} % disable monospaced font for URLs
\hypersetup{
  hidelinks,
  pdfcreator={LaTeX via pandoc}}

\author{}
\date{}

\begin{document}

\hypertarget{ux433ux43bux430ux432ux430-4.-ux441ux43eux437ux43dux430ux43dux438ux435-ux43aux430ux43a-ux43eux43dux442ux43eux43bux43eux433ux438ux447ux435ux441ux43aux430ux44f-ux43fux443ux441ux442ux43eux442ux430}{%
\section{Глава 4. Сознание как онтологическая
Пустота}\label{ux433ux43bux430ux432ux430-4.-ux441ux43eux437ux43dux430ux43dux438ux435-ux43aux430ux43a-ux43eux43dux442ux43eux43bux43eux433ux438ux447ux435ux441ux43aux430ux44f-ux43fux443ux441ux442ux43eux442ux430}}

\hypertarget{ux43fux440ux43eux431ux43bux435ux43cux430-ux43fux435ux440ux432ux43eux43eux441ux43dux43eux432ux430ux43dux438ux44f}{%
\subsection{4.1. Проблема
первооснования}\label{ux43fux440ux43eux431ux43bux435ux43cux430-ux43fux435ux440ux432ux43eux43eux441ux43dux43eux432ux430ux43dux438ux44f}}

Любая онтология неизбежно сталкивается с вопросом первооснования: что
существует \emph{прежде} всего остального и не требует объяснения через
иное.

В классических физических онтологиях таким основанием выступает материя
или поле; в информационных --- информация или вычисление.\\
Однако все подобные основания оказываются проблематичными, поскольку уже
предполагают наличие различий, структур и форм, не объясняя их
происхождения.

Материя всегда дана в виде определённых объектов или состояний.
Информация всегда есть информация \emph{о чём-то}. Математические
структуры предполагают интерпретацию.

Следовательно, ни материя, ни информация, ни математика не могут быть
первооснованием в строгом онтологическом смысле.

В данной теории в качестве первооснования вводится \textbf{Сознание},
однако не в психологическом, феноменологическом или субъективном смысле,
а как \textbf{фундаментальная реальность}, предшествующая любым формам
определённости.

\hypertarget{ux441ux43eux437ux43dux430ux43dux438ux435-ux43dux435-ux44fux432ux43bux44fux435ux442ux441ux44f-ux43eux431ux44aux435ux43aux442ux43eux43c}{%
\subsection{4.2. Сознание не является
объектом}\label{ux441ux43eux437ux43dux430ux43dux438ux435-ux43dux435-ux44fux432ux43bux44fux435ux442ux441ux44f-ux43eux431ux44aux435ux43aux442ux43eux43c}}

Первое и принципиально важное утверждение состоит в том, что Сознание
\textbf{не является объектом}.

Оно не обладает свойствами, не локализовано в пространстве и не
развивается во времени как процесс.

Любая попытка мыслить Сознание как объект немедленно помещает его
\emph{внутрь} некоторой более общей структуры, тем самым лишая его
фундаментального статуса.

Сознание не может быть элементом множества, поскольку само множество уже
есть форма знания.

Сознание также не является процессом: процессы предполагают изменение
состояний во времени, тогда как время в данной онтологии рассматривается
как производное от более фундаментальных актов различения.

Таким образом, Сознание не принадлежит ни к классу вещей, ни к классу
процессов, ни к классу свойств.

\hypertarget{ux43fux43eux43dux44fux442ux438ux435-ux43eux43dux442ux43eux43bux43eux433ux438ux447ux435ux441ux43aux43eux439-ux43fux443ux441ux442ux43eux442ux44b}{%
\subsection{4.3. Понятие онтологической
Пустоты}\label{ux43fux43eux43dux44fux442ux438ux435-ux43eux43dux442ux43eux43bux43eux433ux438ux447ux435ux441ux43aux43eux439-ux43fux443ux441ux442ux43eux442ux44b}}

Чтобы избежать скрытых допущений, Сознание вводится как
\textbf{онтологическая Пустота}.

Под Пустотой здесь понимается не отсутствие бытия и не небытие, а
\textbf{отсутствие определённости}.

Онтологическая Пустота не содержит различий, форм, структур или
отношений, но допускает их возникновение.

Важно подчеркнуть, чем онтологическая Пустота \textbf{не является}:

\begin{itemize}
\item
  она не тождественна физическому вакууму;
\item
  она не является «ничто» в метафизическом смысле;
\item
  она не совпадает с потенциальностью в аристотелевском понимании,
  поскольку не предполагает заранее заданных форм.
\end{itemize}

Онтологическая Пустота --- это не пустота \emph{чего-то}, а отсутствие
самой категории «чего-то».

Контролируемая метафора: Пустота не есть пустой сосуд, ожидающий
наполнения, а отсутствие самого сосуда.

\hypertarget{ux43dux435ux43eux43fux440ux435ux434ux435ux43bux451ux43dux43dux43eux441ux442ux44c-ux43aux430ux43a-ux444ux443ux43dux434ux430ux43cux435ux43dux442ux430ux43bux44cux43dux43eux435-ux441ux43eux441ux442ux43eux44fux43dux438ux435}{%
\subsection{4.4. Неопределённость как фундаментальное
состояние}\label{ux43dux435ux43eux43fux440ux435ux434ux435ux43bux451ux43dux43dux43eux441ux442ux44c-ux43aux430ux43a-ux444ux443ux43dux434ux430ux43cux435ux43dux442ux430ux43bux44cux43dux43eux435-ux441ux43eux441ux442ux43eux44fux43dux438ux435}}

Онтологическая Пустота может быть формально охарактеризована как
состояние \textbf{максимальной неопределённости}.

Неопределённость здесь носит не эпистемический характер (не связана с
недостатком знания наблюдателя), а \textbf{онтологический}:\\
в Пустоте отсутствуют сами основания для различения.

Нет объектов, между которыми можно провести различие. Нет свойств,
которые можно было бы сравнить. Нет отношений, которые можно было бы
зафиксировать.

Таким образом, неопределённость предшествует не только знанию, но и
возможности знания.

Это положение является ключевым для последующей формализации, где
неопределённость будет связана с информационными мерами, однако на
онтологическом уровне она не редуцируется к вероятностному описанию.

\hypertarget{ux43fux43eux447ux435ux43cux443-ux43fux443ux441ux442ux43eux442ux430-ux43dux435ux43eux431ux445ux43eux434ux438ux43cux430}{%
\subsection{4.5. Почему Пустота
необходима}\label{ux43fux43eux447ux435ux43cux443-ux43fux443ux441ux442ux43eux442ux430-ux43dux435ux43eux431ux445ux43eux434ux438ux43cux430}}

Введение онтологической Пустоты не является философской экзотикой, а
логическим следствием стремления к минимальному набору онтологических
допущений.

Если первооснование уже содержит различия, то необходимо объяснить их
происхождение. Если оно уже содержит структуру, необходимо объяснить,
почему именно такую.

Онтологическая Пустота лишена этих проблем, поскольку не содержит
никаких форм, которые требовали бы объяснения.\\
Все формы возникают \emph{из неё}, но не \emph{содержатся в ней
заранее}.

Таким образом, Пустота --- это не гипотеза о «начале Вселенной», а
онтологическое условие возможности любого начала.

\hypertarget{ux441ux43eux437ux43dux430ux43dux438ux435-ux438-ux432ux43eux437ux43cux43eux436ux43dux43eux441ux442ux44c-ux440ux430ux437ux43bux438ux447ux435ux43dux438ux44f}{%
\subsection{4.6. Сознание и возможность
различения}\label{ux441ux43eux437ux43dux430ux43dux438ux435-ux438-ux432ux43eux437ux43cux43eux436ux43dux43eux441ux442ux44c-ux440ux430ux437ux43bux438ux447ux435ux43dux438ux44f}}

Ключевым свойством Сознания, понимаемого как Пустота, является
\textbf{возможность различения}.

Хотя в Пустоте отсутствуют различия, она не является логически замкнутой
или статичной.

Сознание способно к \textbf{самоосознанию}, то есть к различению самого
себя без внешнего источника.

Это положение принципиально отличает предлагаемую онтологию от любых
дуалистических или теистических схем, в которых различие вводится извне.

Различие возникает не потому, что «что-то воздействует на Сознание», а
потому что \textbf{Сознание различает само себя}.

\hypertarget{ux43eux43dux442ux43eux43bux43eux433ux438ux447ux435ux441ux43aux438ux439-ux441ux442ux430ux442ux443ux441-ux441ux430ux43cux43eux43eux441ux43eux437ux43dux430ux43dux438ux44f}{%
\subsection{4.7. Онтологический статус
самоосознания}\label{ux43eux43dux442ux43eux43bux43eux433ux438ux447ux435ux441ux43aux438ux439-ux441ux442ux430ux442ux443ux441-ux441ux430ux43cux43eux43eux441ux43eux437ux43dux430ux43dux438ux44f}}

Самоосознание не является процессом в физическом или психологическом
смысле.\\
Это \textbf{онтологический акт}, в котором впервые возникает
определённость.

Этот акт не подчиняется причинности, поскольку причинность сама является
производной структурой, возникающей лишь после появления устойчивых
различий.

Онтологический акт самоосознания не имеет причины, но имеет следствие:\\
возникновение \textbf{первичного различия}.

С этого момента онтологическая Пустота перестаёт быть абсолютно
неопределённой, и начинается формирование структур знания.

\hypertarget{ux43fux435ux440ux435ux445ux43eux434-ux43eux442-ux43fux443ux441ux442ux43eux442ux44b-ux43a-ux444ux43eux440ux43cux435}{%
\subsection{4.8. Переход от Пустоты к
форме}\label{ux43fux435ux440ux435ux445ux43eux434-ux43eux442-ux43fux443ux441ux442ux43eux442ux44b-ux43a-ux444ux43eux440ux43cux435}}

Возникновение первого различия знаменует переход от Пустоты к форме.\\
Однако этот переход не является скачком от «ничего» к «чему-то».

Пустота не исчезает. Она продолжает присутствовать как фон
неопределённости, на котором существуют и изменяются формы знания.

В этом смысле онтологическая Пустота не принадлежит прошлому --- она
является постоянным аспектом реальности, обеспечивающим возможность
изменений, свободы и новизны.

\hypertarget{ux437ux43dux430ux447ux435ux43dux438ux435-ux434ux43bux44f-ux434ux430ux43bux44cux43dux435ux439ux448ux435ux433ux43e-ux438ux437ux43bux43eux436ux435ux43dux438ux44f}{%
\subsection{4.9. Значение для дальнейшего
изложения}\label{ux437ux43dux430ux447ux435ux43dux438ux435-ux434ux43bux44f-ux434ux430ux43bux44cux43dux435ux439ux448ux435ux433ux43e-ux438ux437ux43bux43eux436ux435ux43dux438ux44f}}

Понимание Сознания как онтологической Пустоты задаёт рамку для всех
последующих глав:

\begin{itemize}
\item
  самодифференциация будет рассмотрена как механизм возникновения
  различий;
\item
  знание --- как форма определённости;
\item
  пространство --- как структура различий;
\item
  время --- как процесс их изменения;
\item
  наблюдатель --- как локализация знания.
\end{itemize}

В следующей главе будет показано, каким образом из онтологической
Пустоты возникает \textbf{самодифференциация}, являющаяся
фундаментальным механизмом порождения мира.

\end{document}
