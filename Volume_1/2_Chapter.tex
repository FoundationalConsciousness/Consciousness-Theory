% Options for packages loaded elsewhere
\PassOptionsToPackage{unicode}{hyperref}
\PassOptionsToPackage{hyphens}{url}
%
\documentclass[
]{article}
\usepackage{amsmath,amssymb}
\usepackage{lmodern}
\usepackage{iftex}
\ifPDFTeX
  \usepackage[T1]{fontenc}
  \usepackage[utf8]{inputenc}
  \usepackage{textcomp} % provide euro and other symbols
\else % if luatex or xetex
  \usepackage{unicode-math}
  \defaultfontfeatures{Scale=MatchLowercase}
  \defaultfontfeatures[\rmfamily]{Ligatures=TeX,Scale=1}
\fi
% Use upquote if available, for straight quotes in verbatim environments
\IfFileExists{upquote.sty}{\usepackage{upquote}}{}
\IfFileExists{microtype.sty}{% use microtype if available
  \usepackage[]{microtype}
  \UseMicrotypeSet[protrusion]{basicmath} % disable protrusion for tt fonts
}{}
\makeatletter
\@ifundefined{KOMAClassName}{% if non-KOMA class
  \IfFileExists{parskip.sty}{%
    \usepackage{parskip}
  }{% else
    \setlength{\parindent}{0pt}
    \setlength{\parskip}{6pt plus 2pt minus 1pt}}
}{% if KOMA class
  \KOMAoptions{parskip=half}}
\makeatother
\usepackage{xcolor}
\setlength{\emergencystretch}{3em} % prevent overfull lines
\providecommand{\tightlist}{%
  \setlength{\itemsep}{0pt}\setlength{\parskip}{0pt}}
\setcounter{secnumdepth}{-\maxdimen} % remove section numbering
\ifLuaTeX
  \usepackage{selnolig}  % disable illegal ligatures
\fi
\IfFileExists{bookmark.sty}{\usepackage{bookmark}}{\usepackage{hyperref}}
\IfFileExists{xurl.sty}{\usepackage{xurl}}{} % add URL line breaks if available
\urlstyle{same} % disable monospaced font for URLs
\hypersetup{
  hidelinks,
  pdfcreator={LaTeX via pandoc}}

\author{}
\date{}

\begin{document}

\hypertarget{ux433ux43bux430ux432ux430-2.-ux43fux440ux435ux434ux435ux43bux44b-ux440ux435ux434ux443ux43aux446ux438ux43eux43dux438ux437ux43cux430-ux438-ux438ux43bux43bux44eux437ux438ux44f-ux43eux431ux44aux44fux441ux43dux435ux43dux438ux44f}{%
\section{Глава 2. Пределы редукционизма и иллюзия
объяснения}\label{ux433ux43bux430ux432ux430-2.-ux43fux440ux435ux434ux435ux43bux44b-ux440ux435ux434ux443ux43aux446ux438ux43eux43dux438ux437ux43cux430-ux438-ux438ux43bux43bux44eux437ux438ux44f-ux43eux431ux44aux44fux441ux43dux435ux43dux438ux44f}}

\hypertarget{ux440ux435ux434ux443ux43aux446ux438ux43eux43dux438ux437ux43c-ux43aux430ux43a-ux43cux435ux442ux43eux434-ux438-ux43aux430ux43a-ux43eux43dux442ux43eux43bux43eux433ux438ux44f}{%
\subsection{2.1. Редукционизм как метод и как
онтология}\label{ux440ux435ux434ux443ux43aux446ux438ux43eux43dux438ux437ux43c-ux43aux430ux43a-ux43cux435ux442ux43eux434-ux438-ux43aux430ux43a-ux43eux43dux442ux43eux43bux43eux433ux438ux44f}}

Редукционизм является одним из самых успешных методов науки. Он
заключается в объяснении сложных явлений через их более простые
компоненты.

Однако важно различать:

\begin{itemize}
\item
  редукционизм как метод исследования;
\item
  редукционизм как онтологическое утверждение.
\end{itemize}

Проблемы возникают тогда, когда метод превращается в онтологию,
утверждающую, что реальность исчерпывается своими элементарными частями.

\hypertarget{ux443ux441ux43fux435ux445ux438-ux438-ux433ux440ux430ux43dux438ux446ux44b-ux440ux435ux434ux443ux43aux446ux438ux43eux43dux438ux437ux43cux430}{%
\subsection{2.2. Успехи и границы
редукционизма}\label{ux443ux441ux43fux435ux445ux438-ux438-ux433ux440ux430ux43dux438ux446ux44b-ux440ux435ux434ux443ux43aux446ux438ux43eux43dux438ux437ux43cux430}}

Редукционизм продемонстрировал выдающиеся успехи:

\begin{itemize}
\item
  в физике элементарных частиц;
\item
  в химии;
\item
  в молекулярной биологии;
\item
  в нейронауках.
\end{itemize}

Однако эти успехи касаются описания структур и механизмов, а не
объяснения самого факта переживания, смысла и опыта.

Редукционизм объясняет \emph{как}, но не объясняет \emph{почему}.

\hypertarget{ux438ux43bux43bux44eux437ux438ux44f-ux43eux431ux44aux44fux441ux43dux435ux43dux438ux44f}{%
\subsection{2.3. Иллюзия
объяснения}\label{ux438ux43bux43bux44eux437ux438ux44f-ux43eux431ux44aux44fux441ux43dux435ux43dux438ux44f}}

Редукционистское объяснение часто создаёт иллюзию понимания. Замена
явления его механизмом не тождественна объяснению явления. Знание всех
нейронных коррелятов боли не объясняет, почему боль переживается как
боль.

Таким образом, редукционизм подменяет онтологическое объяснение
функциональным описанием.

\hypertarget{ux43fux440ux43eux431ux43bux435ux43cux430-ux443ux440ux43eux432ux43dux435ux439-ux43eux43fux438ux441ux430ux43dux438ux44f}{%
\subsection{2.4. Проблема уровней
описания}\label{ux43fux440ux43eux431ux43bux435ux43cux430-ux443ux440ux43eux432ux43dux435ux439-ux43eux43fux438ux441ux430ux43dux438ux44f}}

Реальность обладает несколькими уровнями организации.

Каждый уровень:

\begin{itemize}
\item
  имеет собственные закономерности;
\item
  не сводится полностью к нижележащему;
\item
  требует адекватного языка описания.
\end{itemize}

Переход между уровнями не является логическим выводом, а представляет
собой потерю информации.

Редукция уничтожает структуру,которую пыт ается объяснить.

\hypertarget{ux43dux435ux432ux44bux432ux43eux434ux438ux43cux43eux441ux442ux44c-ux43aux430ux447ux435ux441ux442ux432ux430-ux438ux437-ux43aux43eux43bux438ux447ux435ux441ux442ux432ux430}{%
\subsection{2.5. Невыводимость качества из
количества}\label{ux43dux435ux432ux44bux432ux43eux434ux438ux43cux43eux441ux442ux44c-ux43aux430ux447ux435ux441ux442ux432ux430-ux438ux437-ux43aux43eux43bux438ux447ux435ux441ux442ux432ux430}}

Одной из ключевых проблем редукционизма является невозможность вывести
качественные различия из количественных параметров.

Даже бесконечно точное количественное описание не порождает качества
переживания.

Это указывает на категориальную ошибку: качество и количество относятся
к разным онтологическим уровням.

\hypertarget{ux440ux435ux434ux443ux43aux446ux438ux43eux43dux438ux437ux43c-ux438-ux43fux440ux43eux431ux43bux435ux43cux430-ux441ux43cux44bux441ux43bux430}{%
\subsection{2.6. Редукционизм и проблема
смысла}\label{ux440ux435ux434ux443ux43aux446ux438ux43eux43dux438ux437ux43c-ux438-ux43fux440ux43eux431ux43bux435ux43cux430-ux441ux43cux44bux441ux43bux430}}

Редукционизм принципиально исключает смысл из описания реальности.

Смысл рассматривается как:

\begin{itemize}
\item
  субъективная интерпретация;
\item
  побочный эффект;
\item
  социальная конструкция.
\end{itemize}

Однако без понятия смысла становится невозможным:

\begin{itemize}
\item
  интерпретировать наблюдения;
\item
  различать существенное и случайное;
\item
  обосновывать научные теории.
\end{itemize}

Смысл не может быть редуцирован, поскольку он является условием самого
познания.

\hypertarget{ux441ux430ux43cux43eux440ux435ux444ux435ux440ux435ux43dux442ux43dux43eux435-ux43fux440ux43eux442ux438ux432ux43eux440ux435ux447ux438ux435-ux440ux435ux434ux443ux43aux446ux438ux43eux43dux438ux437ux43cux430}{%
\subsection{2.7. Самореферентное противоречие
редукционизма}\label{ux441ux430ux43cux43eux440ux435ux444ux435ux440ux435ux43dux442ux43dux43eux435-ux43fux440ux43eux442ux438ux432ux43eux440ux435ux447ux438ux435-ux440ux435ux434ux443ux43aux446ux438ux43eux43dux438ux437ux43cux430}}

Редукционизм сталкивается с самореферентным противоречием.

Если всё является следствием физических процессов, то и сама теория
редукционизма является продуктом этих процессов, а не носителем истины.

Это подрывает эпистемологический статус самой редукционистской позиции.

\hypertarget{ux43dux435ux43eux431ux445ux43eux434ux438ux43cux43eux441ux442ux44c-ux43eux43dux442ux43eux43bux43eux433ux438ux447ux435ux441ux43aux43eux433ux43e-ux440ux430ux441ux448ux438ux440ux435ux43dux438ux44f}{%
\subsection{2.8. Необходимость онтологического
расширения}\label{ux43dux435ux43eux431ux445ux43eux434ux438ux43cux43eux441ux442ux44c-ux43eux43dux442ux43eux43bux43eux433ux438ux447ux435ux441ux43aux43eux433ux43e-ux440ux430ux441ux448ux438ux440ux435ux43dux438ux44f}}

Отказ от редукционизма не означает отказа от науки.

Он означает признание того, что:

\begin{itemize}
\item
  реальность не исчерпывается физическими описаниями;
\item
  существуют фундаментальные уровни, не сводимые друг к другу;
\item
  требуется более глубокая онтология.
\end{itemize}

Сознание не может быть объяснено через редукцию,поскольку оно является\\
условием любого объяснения.

\hypertarget{ux43fux43eux434ux433ux43eux442ux43eux432ux43aux430-ux43a-ux43dux43eux432ux43eux439-ux43eux43dux442ux43eux43bux43eux433ux438ux438}{%
\subsection{2.9. Подготовка к новой
онтологии}\label{ux43fux43eux434ux433ux43eux442ux43eux432ux43aux430-ux43a-ux43dux43eux432ux43eux439-ux43eux43dux442ux43eux43bux43eux433ux438ux438}}

Пределы редукционизма не являются поражением науки.

Они указывают на необходимость смены основания.

Если редукция не объясняет, то требуется онтология, в которой:

\begin{itemize}
\item
  сознание первично;
\item
  знание является формой бытия;
\item
  различие --- источником структуры.
\end{itemize}

Эта онтология будет развёрнута в следующих главах.

\end{document}
