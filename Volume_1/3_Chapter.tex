% Options for packages loaded elsewhere
\PassOptionsToPackage{unicode}{hyperref}
\PassOptionsToPackage{hyphens}{url}
%
\documentclass[
]{article}
\usepackage{amsmath,amssymb}
\usepackage{lmodern}
\usepackage{iftex}
\ifPDFTeX
  \usepackage[T1]{fontenc}
  \usepackage[utf8]{inputenc}
  \usepackage{textcomp} % provide euro and other symbols
\else % if luatex or xetex
  \usepackage{unicode-math}
  \defaultfontfeatures{Scale=MatchLowercase}
  \defaultfontfeatures[\rmfamily]{Ligatures=TeX,Scale=1}
\fi
% Use upquote if available, for straight quotes in verbatim environments
\IfFileExists{upquote.sty}{\usepackage{upquote}}{}
\IfFileExists{microtype.sty}{% use microtype if available
  \usepackage[]{microtype}
  \UseMicrotypeSet[protrusion]{basicmath} % disable protrusion for tt fonts
}{}
\makeatletter
\@ifundefined{KOMAClassName}{% if non-KOMA class
  \IfFileExists{parskip.sty}{%
    \usepackage{parskip}
  }{% else
    \setlength{\parindent}{0pt}
    \setlength{\parskip}{6pt plus 2pt minus 1pt}}
}{% if KOMA class
  \KOMAoptions{parskip=half}}
\makeatother
\usepackage{xcolor}
\setlength{\emergencystretch}{3em} % prevent overfull lines
\providecommand{\tightlist}{%
  \setlength{\itemsep}{0pt}\setlength{\parskip}{0pt}}
\setcounter{secnumdepth}{-\maxdimen} % remove section numbering
\ifLuaTeX
  \usepackage{selnolig}  % disable illegal ligatures
\fi
\IfFileExists{bookmark.sty}{\usepackage{bookmark}}{\usepackage{hyperref}}
\IfFileExists{xurl.sty}{\usepackage{xurl}}{} % add URL line breaks if available
\urlstyle{same} % disable monospaced font for URLs
\hypersetup{
  hidelinks,
  pdfcreator={LaTeX via pandoc}}

\author{}
\date{}

\begin{document}

\hypertarget{ux433ux43bux430ux432ux430-3.-ux43aux440ux438ux437ux438ux441-ux43eux431ux44aux435ux43aux442ux438ux432ux43dux43eux441ux442ux438-ux438-ux440ux43eux43bux44c-ux43dux430ux431ux43bux44eux434ux430ux442ux435ux43bux44f}{%
\section{Глава 3. Кризис объективности и роль
наблюдателя}\label{ux433ux43bux430ux432ux430-3.-ux43aux440ux438ux437ux438ux441-ux43eux431ux44aux435ux43aux442ux438ux432ux43dux43eux441ux442ux438-ux438-ux440ux43eux43bux44c-ux43dux430ux431ux43bux44eux434ux430ux442ux435ux43bux44f}}

\hypertarget{ux43aux43bux430ux441ux441ux438ux447ux435ux441ux43aux438ux439-ux438ux434ux435ux430ux43b-ux43eux431ux44aux435ux43aux442ux438ux432ux43dux43eux441ux442ux438}{%
\subsection{3.1. Классический идеал
объективности}\label{ux43aux43bux430ux441ux441ux438ux447ux435ux441ux43aux438ux439-ux438ux434ux435ux430ux43b-ux43eux431ux44aux435ux43aux442ux438ux432ux43dux43eux441ux442ux438}}

Классическая наука строилась на идеале объективности, под которым
понималось описание мира, независимое от наблюдателя.

Предполагалось, что:

\begin{itemize}
\item
  реальность существует «сама по себе»;
\item
  наблюдатель лишь фиксирует факты;
\item
  знание является зеркалом внешнего мира.
\end{itemize}

Этот идеал можно назвать «взглядом ниоткуда».

\hypertarget{ux438ux43bux43bux44eux437ux438ux44f-ux432ux437ux433ux43bux44fux434ux430-ux43dux438ux43eux442ux43aux443ux434ux430}{%
\subsection{3.2. Иллюзия взгляда
ниоткуда}\label{ux438ux43bux43bux44eux437ux438ux44f-ux432ux437ux433ux43bux44fux434ux430-ux43dux438ux43eux442ux43aux443ux434ux430}}

«Взгляд ниоткуда» предполагает существование абсолютной точки зрения, не
включённой в реальность.

Однако любое знание:

\begin{itemize}
\item
  формулируется в языке;
\item
  использует понятия;
\item
  предполагает различение;
\end{itemize}

осуществляется из некоторой позиции.

Наблюдение без наблюдателя является логическим противоречием.

Объективность в строгом смысле оказывается недостижимым идеалом, а не
онтологическим фактом.

\hypertarget{ux43dux430ux431ux43bux44eux434ux430ux442ux435ux43bux44c-ux432-ux43aux43bux430ux441ux441ux438ux447ux435ux441ux43aux43eux439-ux444ux438ux437ux438ux43aux435}{%
\subsection{3.3. Наблюдатель в классической
физике}\label{ux43dux430ux431ux43bux44eux434ux430ux442ux435ux43bux44c-ux432-ux43aux43bux430ux441ux441ux438ux447ux435ux441ux43aux43eux439-ux444ux438ux437ux438ux43aux435}}

В классической физике роль наблюдателя минимизируется.

Измерение считается:

\begin{itemize}
\item
  пассивным;
\item
  не влияющим на объект;
\item
  не изменяющим его состояние.
\end{itemize}

Однако это предположение является не выводом, а постулатом.

Оно работает лишь в пределах определённого масштаба и точности.

\hypertarget{ux43aux432ux430ux43dux442ux43eux432ux44bux439-ux43aux440ux438ux437ux438ux441-ux43eux431ux44aux435ux43aux442ux438ux432ux43dux43eux441ux442ux438}{%
\subsection{3.4. Квантовый кризис
объективности}\label{ux43aux432ux430ux43dux442ux43eux432ux44bux439-ux43aux440ux438ux437ux438ux441-ux43eux431ux44aux435ux43aux442ux438ux432ux43dux43eux441ux442ux438}}

В квантовой теории идеал объективности рушится.

Результат измерения:

\begin{itemize}
\item
  зависит от выбора наблюдаемой величины;
\item
  не существует до акта измерения;
\item
  является не выявлением, а актуализацией состояния.
\end{itemize}

Наблюдатель перестаёт быть внешним элементом системы.

Он становится частью самого физического описания.

\hypertarget{ux43fux430ux440ux430ux434ux43eux43aux441-ux43dux430ux431ux43bux44eux434ux430ux442ux435ux43bux44f}{%
\subsection{3.5. Парадокс
наблюдателя}\label{ux43fux430ux440ux430ux434ux43eux43aux441-ux43dux430ux431ux43bux44eux434ux430ux442ux435ux43bux44f}}

Квантовая механика вводит фундаментальный парадокс:

Чтобы описать систему, необходимо наблюдение, но наблюдение изменяет
систему.

Таким образом:

\begin{itemize}
\item
  объект без наблюдения не определён;
\item
  наблюдение без объекта бессмысленно.
\end{itemize}

Объективность оказывается внутренне противоречивой.

\hypertarget{ux440ux430ux441ux448ux438ux440ux435ux43dux438ux435-ux43aux440ux438ux437ux438ux441ux430-ux437ux430-ux43fux440ux435ux434ux435ux43bux44b-ux444ux438ux437ux438ux43aux438}{%
\subsection{3.6. Расширение кризиса за пределы
физики}\label{ux440ux430ux441ux448ux438ux440ux435ux43dux438ux435-ux43aux440ux438ux437ux438ux441ux430-ux437ux430-ux43fux440ux435ux434ux435ux43bux44b-ux444ux438ux437ux438ux43aux438}}

Кризис объективности не ограничивается физикой.

В биологии:

\begin{itemize}
\item
  наблюдение влияет на поведение систем;
\item
  интерпретация данных зависит от модели.
\end{itemize}

В нейронауках:

\begin{itemize}
\item
  сознание изучается средствами сознания;
\item
  объяснение опирается на то, что должно быть объяснено.
\end{itemize}

В философии науки:

\begin{itemize}
\item
  теория предшествует факту;
\item
  данные не существуют вне интерпретации.
\end{itemize}

\hypertarget{ux43eux431ux44aux435ux43aux442ux438ux432ux43dux43eux441ux442ux44c-ux43aux430ux43a-ux438ux43dux442ux435ux440ux441ux443ux431ux44aux435ux43aux442ux438ux432ux43dux43eux441ux442ux44c}{%
\subsection{3.7. Объективность как
интерсубъективность}\label{ux43eux431ux44aux435ux43aux442ux438ux432ux43dux43eux441ux442ux44c-ux43aux430ux43a-ux438ux43dux442ux435ux440ux441ux443ux431ux44aux435ux43aux442ux438ux432ux43dux43eux441ux442ux44c}}

Современное понимание объективности смещается от абсолютности к
интерсубъективной устойчивости.

Объективным считается то, что:

\begin{itemize}
\item
  воспроизводимо;
\item
  согласовано между наблюдателями;
\item
  устойчиво к изменению перспектив.
\end{itemize}

Однако это уже не «независимость», а согласование внутри системы
наблюдателей.

Объективность становится производной от сознания, а не альтернативой
ему.

\hypertarget{ux43dux430ux431ux43bux44eux434ux430ux442ux435ux43bux44c-ux43aux430ux43a-ux443ux441ux43bux43eux432ux438ux435-ux440ux435ux430ux43bux44cux43dux43eux441ux442ux438}{%
\subsection{3.8. Наблюдатель как условие
реальности}\label{ux43dux430ux431ux43bux44eux434ux430ux442ux435ux43bux44c-ux43aux430ux43a-ux443ux441ux43bux43eux432ux438ux435-ux440ux435ux430ux43bux44cux43dux43eux441ux442ux438}}

Если:

\begin{itemize}
\item
  любое знание предполагает наблюдателя;
\item
  любое описание осуществляется из позиции;
\item
  любое различие требует акта различения,
\end{itemize}

то наблюдатель не может быть вторичным элементом мира.

Он является условием проявления реальности как таковой.

Реальность без наблюдателя не описуема, не различима и не определима.

\hypertarget{ux43eux442-ux44dux43fux438ux441ux442ux435ux43cux43eux43bux43eux433ux438ux438-ux43a-ux43eux43dux442ux43eux43bux43eux433ux438ux438}{%
\subsection{3.9. От эпистемологии к
онтологии}\label{ux43eux442-ux44dux43fux438ux441ux442ux435ux43cux43eux43bux43eux433ux438ux438-ux43a-ux43eux43dux442ux43eux43bux43eux433ux438ux438}}

Попытка свести роль наблюдателя к эпистемологии не решает проблему.

Если наблюдатель необходим для возникновения фактов, то он не просто
познаёт мир ---\\
он участвует в его актуализации.

Следовательно, наблюдатель должен быть онтологическим элементом теории.

\hypertarget{ux43fux43eux434ux433ux43eux442ux43eux432ux43aux430-ux43a-ux432ux432ux435ux434ux435ux43dux438ux44e-ux441ux43eux437ux43dux430ux43dux438ux44f}{%
\subsection{3.10. Подготовка к введению
Сознания}\label{ux43fux43eux434ux433ux43eux442ux43eux432ux43aux430-ux43a-ux432ux432ux435ux434ux435ux43dux438ux44e-ux441ux43eux437ux43dux430ux43dux438ux44f}}

Наблюдатель --- это не обязательно человек, мозг или субъект в
психологическом смысле.

Это:

\begin{itemize}
\item
  способность различать;
\item
  способность актуализировать состояние;
\item
  способность удерживать знание.
\end{itemize}

Эти свойства указывают на более фундаментальное основание, чем
физические объекты.

Таким основанием может быть только Сознание.

\hypertarget{ux432ux44bux432ux43eux434ux44b-ux433ux43bux430ux432ux44b}{%
\subsection{3.11. Выводы
главы}\label{ux432ux44bux432ux43eux434ux44b-ux433ux43bux430ux432ux44b}}

\begin{enumerate}
\def\labelenumi{\arabic{enumi}.}
\item
  Абсолютная объективность невозможна.
\item
  «Взгляд ниоткуда» --- логическая фикция.
\item
  Наблюдатель является необходимым элементом описания.
\item
  Объективность производна от интерсубъективности.
\item
  Наблюдатель должен быть введён онтологически.
\end{enumerate}

\hypertarget{ux43fux435ux440ux435ux445ux43eux434-ux43a-ux441ux43bux435ux434ux443ux44eux449ux435ux439-ux433ux43bux430ux432ux435}{%
\subsection{Переход к следующей
главе}\label{ux43fux435ux440ux435ux445ux43eux434-ux43a-ux441ux43bux435ux434ux443ux44eux449ux435ux439-ux433ux43bux430ux432ux435}}

Кризис объективности делает неизбежным пересмотр основания реальности.

Если наблюдатель неустраним, то реальность не может быть внешней по
отношению к нему.

Следующий шаг --- признать Сознание не как объект в мире, а как
основание мира.

\end{document}
