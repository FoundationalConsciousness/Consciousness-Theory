% Options for packages loaded elsewhere
\PassOptionsToPackage{unicode}{hyperref}
\PassOptionsToPackage{hyphens}{url}
%
\documentclass[
]{article}
\usepackage{amsmath,amssymb}
\usepackage{lmodern}
\usepackage{iftex}
\ifPDFTeX
  \usepackage[T1]{fontenc}
  \usepackage[utf8]{inputenc}
  \usepackage{textcomp} % provide euro and other symbols
\else % if luatex or xetex
  \usepackage{unicode-math}
  \defaultfontfeatures{Scale=MatchLowercase}
  \defaultfontfeatures[\rmfamily]{Ligatures=TeX,Scale=1}
\fi
% Use upquote if available, for straight quotes in verbatim environments
\IfFileExists{upquote.sty}{\usepackage{upquote}}{}
\IfFileExists{microtype.sty}{% use microtype if available
  \usepackage[]{microtype}
  \UseMicrotypeSet[protrusion]{basicmath} % disable protrusion for tt fonts
}{}
\makeatletter
\@ifundefined{KOMAClassName}{% if non-KOMA class
  \IfFileExists{parskip.sty}{%
    \usepackage{parskip}
  }{% else
    \setlength{\parindent}{0pt}
    \setlength{\parskip}{6pt plus 2pt minus 1pt}}
}{% if KOMA class
  \KOMAoptions{parskip=half}}
\makeatother
\usepackage{xcolor}
\setlength{\emergencystretch}{3em} % prevent overfull lines
\providecommand{\tightlist}{%
  \setlength{\itemsep}{0pt}\setlength{\parskip}{0pt}}
\setcounter{secnumdepth}{-\maxdimen} % remove section numbering
\ifLuaTeX
  \usepackage{selnolig}  % disable illegal ligatures
\fi
\IfFileExists{bookmark.sty}{\usepackage{bookmark}}{\usepackage{hyperref}}
\IfFileExists{xurl.sty}{\usepackage{xurl}}{} % add URL line breaks if available
\urlstyle{same} % disable monospaced font for URLs
\hypersetup{
  hidelinks,
  pdfcreator={LaTeX via pandoc}}

\author{}
\date{}

\begin{document}

\hypertarget{ux433ux43bux430ux432ux430-5.-ux441ux430ux43cux43eux43eux441ux43eux437ux43dux430ux43dux438ux435-ux438-ux441ux430ux43cux43eux434ux438ux444ux444ux435ux440ux435ux43dux446ux438ux430ux446ux438ux44f}{%
\section{Глава 5. Самоосознание и
самодифференциация}\label{ux433ux43bux430ux432ux430-5.-ux441ux430ux43cux43eux43eux441ux43eux437ux43dux430ux43dux438ux435-ux438-ux441ux430ux43cux43eux434ux438ux444ux444ux435ux440ux435ux43dux446ux438ux430ux446ux438ux44f}}

\hypertarget{ux43dux435ux43eux431ux445ux43eux434ux438ux43cux43eux441ux442ux44c-ux43fux435ux440ux435ux445ux43eux434ux430-ux43eux442-ux43fux443ux441ux442ux43eux442ux44b-ux43a-ux440ux430ux437ux43bux438ux447ux438ux44e}{%
\subsection{5.1. Необходимость перехода от Пустоты к
различию}\label{ux43dux435ux43eux431ux445ux43eux434ux438ux43cux43eux441ux442ux44c-ux43fux435ux440ux435ux445ux43eux434ux430-ux43eux442-ux43fux443ux441ux442ux43eux442ux44b-ux43a-ux440ux430ux437ux43bux438ux447ux438ux44e}}

В предыдущей главе Сознание было введено как онтологическая Пустота ---
состояние отсутствия определённости, в котором не заданы ни объекты, ни
свойства, ни отношения.

Однако онтология, ограничивающаяся лишь Пустотой, была бы статичной и не
объясняла бы возникновения форм, структур и наблюдаемого мира.

Следовательно, необходимо ввести механизм, посредством которого из
онтологической Пустоты возникает определённость, не прибегая к внешним
причинам или дополнительным сущностям.

Таким механизмом является \textbf{самоосознание Сознания}, приводящее к
\textbf{самодифференциации}.

\hypertarget{ux441ux430ux43cux43eux43eux441ux43eux437ux43dux430ux43dux438ux435-ux43aux430ux43a-ux43eux43dux442ux43eux43bux43eux433ux438ux447ux435ux441ux43aux430ux44f-ux441ux43fux43eux441ux43eux431ux43dux43eux441ux442ux44c}{%
\subsection{5.2. Самоосознание как онтологическая
способность}\label{ux441ux430ux43cux43eux43eux441ux43eux437ux43dux430ux43dux438ux435-ux43aux430ux43a-ux43eux43dux442ux43eux43bux43eux433ux438ux447ux435ux441ux43aux430ux44f-ux441ux43fux43eux441ux43eux431ux43dux43eux441ux442ux44c}}

Самоосознание в данной теории не следует понимать в психологическом или
рефлексивном смысле. Речь идёт не о субъективном переживании и не о
наличии «я», а о фундаментальной способности Сознания \textbf{различать
само себя}.

Самоосознание не предполагает предварительного разделения на субъект и
объект. Напротив, именно акт самоосознания впервые порождает различие,
на основе которого впоследствии могут возникнуть любые формы
субъективности и объективности.

Таким образом, самоосознание является не свойством некоторой сущности, а
\textbf{первичным онтологическим актом}.

\hypertarget{ux441ux430ux43cux43eux434ux438ux444ux444ux435ux440ux435ux43dux446ux438ux430ux446ux438ux44f-ux43aux430ux43a-ux432ux43eux437ux43dux438ux43aux43dux43eux432ux435ux43dux438ux435-ux440ux430ux437ux43bux438ux447ux438ux44f}{%
\subsection{5.3. Самодифференциация как возникновение
различия}\label{ux441ux430ux43cux43eux434ux438ux444ux444ux435ux440ux435ux43dux446ux438ux430ux446ux438ux44f-ux43aux430ux43a-ux432ux43eux437ux43dux438ux43aux43dux43eux432ux435ux43dux438ux435-ux440ux430ux437ux43bux438ux447ux438ux44f}}

\textbf{Самодифференциация} --- это процесс, в результате которого
Сознание, оставаясь единым, порождает внутри себя различия.

Важно подчеркнуть: самодифференциация не является делением Сознания на
части. Сознание не фрагментируется и не утрачивает своей целостности.

Различия возникают не \emph{между} частями, а \emph{внутри} единого поля
Сознания.

Контролируемая метафора: самодифференциация подобна появлению узора на
однородной поверхности, не разрушающему саму поверхность.

\hypertarget{ux43fux435ux440ux432ux438ux447ux43dux43eux435-ux440ux430ux437ux43bux438ux447ux438ux435}{%
\subsection{5.4. Первичное
различие}\label{ux43fux435ux440ux432ux438ux447ux43dux43eux435-ux440ux430ux437ux43bux438ux447ux438ux435}}

Результатом акта самодифференциации является \textbf{первичное
различие}.

Это различие:

\begin{itemize}
\item
  не является пространственным;
\item
  не является количественным;
\item
  не предполагает заранее заданных категорий.
\end{itemize}

Первичное различие --- это минимальная форма определённости, позволяющая
говорить о «чём-то» в онтологическом смысле.

До возникновения первичного различия невозможно говорить ни о состоянии,
ни о процессе, ни о событии.

\hypertarget{ux43eux43dux442ux43eux43bux43eux433ux438ux447ux435ux441ux43aux438ux439-ux441ux442ux430ux442ux443ux441-ux430ux43aux442ux430}{%
\subsection{5.5. Онтологический статус
акта}\label{ux43eux43dux442ux43eux43bux43eux433ux438ux447ux435ux441ux43aux438ux439-ux441ux442ux430ux442ux443ux441-ux430ux43aux442ux430}}

Акт самодифференциации является \textbf{онтологическим актом}, а не
событием в физическом времени.

Он:

\begin{itemize}
\item
  не имеет причины в классическом смысле;
\item
  не может быть выведен из предшествующего состояния, поскольку до него
  отсутствует структура состояний;
\item
  не подчиняется законам, которые сами возникают лишь после появления
  различий.
\end{itemize}

В этом смысле онтологический акт является первичным по отношению к любой
причинности и динамике.

\hypertarget{ux43cux43dux43eux436ux435ux441ux442ux432ux435ux43dux43dux43eux441ux442ux44c-ux440ux430ux437ux43bux438ux447ux438ux439-ux438-ux43dux430ux447ux430ux43bux43e-ux441ux442ux440ux443ux43aux442ux443ux440ux44b}{%
\subsection{5.6. Множественность различий и начало
структуры}\label{ux43cux43dux43eux436ux435ux441ux442ux432ux435ux43dux43dux43eux441ux442ux44c-ux440ux430ux437ux43bux438ux447ux438ux439-ux438-ux43dux430ux447ux430ux43bux43e-ux441ux442ux440ux443ux43aux442ux443ux440ux44b}}

Первичное различие не остаётся изолированным. Однажды возникнув,
различие создаёт возможность для последующих различений.

Сознание, различив само себя один раз, получает способность различать
себя снова, формируя \textbf{множественность различий}.

Эта множественность ещё не является пространством или временем, но уже
представляет собой \textbf{зачаток структуры}.

Структура возникает не как наложенная схема, а как сеть взаимосвязанных
различий.

\hypertarget{ux43eux442-ux440ux430ux437ux43bux438ux447ux438ux44f-ux43a-ux437ux43dux430ux43dux438ux44e}{%
\subsection{5.7. От различия к
знанию}\label{ux43eux442-ux440ux430ux437ux43bux438ux447ux438ux44f-ux43a-ux437ux43dux430ux43dux438ux44e}}

Устойчивые конфигурации различий формируют \textbf{знание}.

Знание в данном контексте не тождественно осведомлённости или
представлению. Это онтологическая форма определённости, возникающая
внутри Сознания.

Таким образом:

\begin{itemize}
\item
  различие --- элементарная форма определённости;
\item
  знание --- устойчивая конфигурация различий.
\end{itemize}

Эта линия будет развита в следующей главе, где знание будет рассмотрено
как форма.

\hypertarget{ux441ux430ux43cux43eux434ux438ux444ux444ux435ux440ux435ux43dux446ux438ux430ux446ux438ux44f-ux438-ux432ux43eux437ux43cux43eux436ux43dux43eux441ux442ux44c-ux444ux43eux440ux43cux430ux43bux438ux437ux430ux446ux438ux438}{%
\subsection{5.8. Самодифференциация и возможность
формализации}\label{ux441ux430ux43cux43eux434ux438ux444ux444ux435ux440ux435ux43dux446ux438ux430ux446ux438ux44f-ux438-ux432ux43eux437ux43cux43eux436ux43dux43eux441ux442ux44c-ux444ux43eux440ux43cux430ux43bux438ux437ux430ux446ux438ux438}}

Самодифференциация подготавливает переход к математической формализации.

Различия допускают:

\begin{itemize}
\item
  сравнение;
\item
  измерение;
\item
  введение меры.
\end{itemize}

В Томе II самодифференциация будет представлена в виде оператора,
действующего в пространстве состояний Сознания, а различия --- как
элементы, между которыми можно ввести информационные метрики.

Однако на онтологическом уровне важно подчеркнуть, что формализация
\textbf{описывает}, но не порождает сам процесс самодифференциации.

\hypertarget{ux441ux432ux44fux437ux44c-ux441-ux43dux430ux431ux43bux44eux434ux435ux43dux438ux435ux43c}{%
\subsection{5.9. Связь с
наблюдением}\label{ux441ux432ux44fux437ux44c-ux441-ux43dux430ux431ux43bux44eux434ux435ux43dux438ux435ux43c}}

Хотя наблюдатель как особый режим знания будет рассмотрен позднее, уже
на данном этапе можно указать на принципиальную связь между
самодифференциацией и наблюдением.

Любое наблюдение является частным случаем самодифференциации, в котором
выбор различия локализуется.

Таким образом, наблюдение не вводит различие, а актуализирует уже
возможные различия в структуре Сознания.

\hypertarget{ux437ux43dux430ux447ux435ux43dux438ux435-ux433ux43bux430ux432ux44b}{%
\subsection{5.10. Значение
главы}\label{ux437ux43dux430ux447ux435ux43dux438ux435-ux433ux43bux430ux432ux44b}}

В этой главе был введён фундаментальный механизм порождения
определённости:

\begin{itemize}
\item
  самоосознание как способность Сознания различать само себя;
\item
  самодифференциация как онтологический акт;
\item
  различие как первичная структура.
\end{itemize}

Тем самым создано основание для:

\begin{itemize}
\item
  введения знания как формы;
\item
  построения пространства различий;
\item
  описания времени как процесса изменений.
\end{itemize}

В следующей главе будет показано, каким образом из различий формируются
\textbf{формы знания}, обладающие устойчивостью и повторяемостью.

\end{document}
