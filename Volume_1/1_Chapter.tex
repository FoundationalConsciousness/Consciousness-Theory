% Options for packages loaded elsewhere
\PassOptionsToPackage{unicode}{hyperref}
\PassOptionsToPackage{hyphens}{url}
%
\documentclass[
]{article}
\usepackage{amsmath,amssymb}
\usepackage{lmodern}
\usepackage{iftex}
\ifPDFTeX
  \usepackage[T1]{fontenc}
  \usepackage[utf8]{inputenc}
  \usepackage{textcomp} % provide euro and other symbols
\else % if luatex or xetex
  \usepackage{unicode-math}
  \defaultfontfeatures{Scale=MatchLowercase}
  \defaultfontfeatures[\rmfamily]{Ligatures=TeX,Scale=1}
\fi
% Use upquote if available, for straight quotes in verbatim environments
\IfFileExists{upquote.sty}{\usepackage{upquote}}{}
\IfFileExists{microtype.sty}{% use microtype if available
  \usepackage[]{microtype}
  \UseMicrotypeSet[protrusion]{basicmath} % disable protrusion for tt fonts
}{}
\makeatletter
\@ifundefined{KOMAClassName}{% if non-KOMA class
  \IfFileExists{parskip.sty}{%
    \usepackage{parskip}
  }{% else
    \setlength{\parindent}{0pt}
    \setlength{\parskip}{6pt plus 2pt minus 1pt}}
}{% if KOMA class
  \KOMAoptions{parskip=half}}
\makeatother
\usepackage{xcolor}
\setlength{\emergencystretch}{3em} % prevent overfull lines
\providecommand{\tightlist}{%
  \setlength{\itemsep}{0pt}\setlength{\parskip}{0pt}}
\setcounter{secnumdepth}{-\maxdimen} % remove section numbering
\ifLuaTeX
  \usepackage{selnolig}  % disable illegal ligatures
\fi
\IfFileExists{bookmark.sty}{\usepackage{bookmark}}{\usepackage{hyperref}}
\IfFileExists{xurl.sty}{\usepackage{xurl}}{} % add URL line breaks if available
\urlstyle{same} % disable monospaced font for URLs
\hypersetup{
  hidelinks,
  pdfcreator={LaTeX via pandoc}}

\author{}
\date{}

\begin{document}

\hypertarget{ux433ux43bux430ux432ux430-1.-ux43eux43dux442ux43eux43bux43eux433ux438ux447ux435ux441ux43aux438ux439-ux442ux443ux43fux438ux43a-ux444ux438ux437ux438ux43aux430ux43bux438ux437ux43cux430}{%
\section{Глава 1. Онтологический тупик
физикализма}\label{ux433ux43bux430ux432ux430-1.-ux43eux43dux442ux43eux43bux43eux433ux438ux447ux435ux441ux43aux438ux439-ux442ux443ux43fux438ux43a-ux444ux438ux437ux438ux43aux430ux43bux438ux437ux43cux430}}

\hypertarget{ux434ux43eux43cux438ux43dux438ux440ux443ux44eux449ux430ux44f-ux43eux43dux442ux43eux43bux43eux433ux438ux44f-ux441ux43eux432ux440ux435ux43cux435ux43dux43dux43eux439-ux43dux430ux443ux43aux438}{%
\subsection{1.1. Доминирующая онтология современной
науки}\label{ux434ux43eux43cux438ux43dux438ux440ux443ux44eux449ux430ux44f-ux43eux43dux442ux43eux43bux43eux433ux438ux44f-ux441ux43eux432ux440ux435ux43cux435ux43dux43dux43eux439-ux43dux430ux443ux43aux438}}

Современная научная картина мира в своей основе опирается на физикализм
--- онтологическую позицию, согласно которой фундаментальной реальностью
является материя и её физические законы.

В этой картине:

материя считается первичной;

сознание --- вторичным;

наблюдение --- производным процессом;

смысл и ценность --- нефундаментальными.

Физикализм не всегда формулируется явно, но он присутствует как неявное
допущение в большинстве научных теорий.

\hypertarget{ux43cux430ux442ux435ux440ux438ux44f-ux43aux430ux43a-ux43fux43eux441ux442ux443ux43bux430ux442-ux430-ux43dux435-ux432ux44bux432ux43eux434}{%
\subsection{1.2. Материя как постулат, а не
вывод}\label{ux43cux430ux442ux435ux440ux438ux44f-ux43aux430ux43a-ux43fux43eux441ux442ux443ux43bux430ux442-ux430-ux43dux435-ux432ux44bux432ux43eux434}}

Ключевой проблемой физикализма является то, что материя в нём
\textbf{постулируется}, а не выводится.

Физические теории описывают: структуры, взаимодействия, законы,
симметрии, но не отвечают на вопрос, \emph{почему} вообще существует
нечто, обладающее физическими свойствами.

Понятие материи вводится как исходная данность, а не как результат более
глубокой онтологии.

Таким образом, физикализм опирается на нефундированный онтологический
постулат.

\hypertarget{ux43fux440ux43eux431ux43bux435ux43cux430-ux441ux43eux437ux43dux430ux43dux438ux44f-ux432-ux444ux438ux437ux438ux43aux430ux43bux438ux437ux43cux435}{%
\subsection{1.3. Проблема сознания в
физикализме}\label{ux43fux440ux43eux431ux43bux435ux43cux430-ux441ux43eux437ux43dux430ux43dux438ux44f-ux432-ux444ux438ux437ux438ux43aux430ux43bux438ux437ux43cux435}}

Сознание в физикалистской картине мира рассматривается как:

\begin{itemize}
\item
  продукт нейронной активности;
\item
  функция сложных физических систем;
\item
  эпифеномен материальных процессов.
\end{itemize}

Однако ни одна из этих позиций не объясняет, каким образом субъективный
опыт может быть выведен из объективных физических описаний.

Даже полное знание о физических процессах в мозге не позволяет логически
вывести наличие переживания как такового.

Это указывает на принципиальную невыводимость сознания из физического
описания.

\hypertarget{ux44dux43fux438ux444ux435ux43dux43eux43cux435ux43dux430ux43bux438ux437ux43c-ux438-ux435ux433ux43e-ux43fux440ux43eux442ux438ux432ux43eux440ux435ux447ux438ux44f}{%
\subsection{1.4. Эпифеноменализм и его
противоречия}\label{ux44dux43fux438ux444ux435ux43dux43eux43cux435ux43dux430ux43bux438ux437ux43c-ux438-ux435ux433ux43e-ux43fux440ux43eux442ux438ux432ux43eux440ux435ux447ux438ux44f}}

Попытка сохранить физикализм часто приводит к эпифеноменализму ---
позиции, согласно которой сознание существует, но не обладает каузальной
силой.

Однако эпифеноменализм сталкивается с рядом фундаментальных
противоречий:

\begin{itemize}
\item
  если сознание не влияет на физические процессы, то невозможно
  объяснить\\
  его эволюционную устойчивость;
\item
  если сознание не имеет каузального статуса, то само утверждение о его
  существовании теряет основание;
\item
  научное исследование сознания становится бессмысленным, поскольку
  объект исследования не оказывает влияния на наблюдаемую реальность.
\end{itemize}

Эпифеноменализм логически неустойчив.

\hypertarget{ux43fux440ux43eux431ux43bux435ux43cux430-ux43dux430ux431ux43bux44eux434ux430ux442ux435ux43bux44f}{%
\subsection{1.5. Проблема
наблюдателя}\label{ux43fux440ux43eux431ux43bux435ux43cux430-ux43dux430ux431ux43bux44eux434ux430ux442ux435ux43bux44f}}

В физикализме наблюдатель рассматривается как внешняя или вторичная
сущность.

Однако в современной физике роль наблюдателя оказывается принципиальной:

\begin{itemize}
\item
  в квантовой механике;
\item
  в теории измерения;
\item
  в статистической интерпретации состояний.
\end{itemize}

Наблюдение не может быть полностью исключено из описания, поскольку
именно оно определяет, какая из возможных реальностей становится
актуальной.

Это приводит к парадоксу: наблюдатель необходим для описания мира, но не
имеет онтологического статуса.

\hypertarget{ux43bux43eux433ux438ux447ux435ux441ux43aux438ux439-ux442ux443ux43fux438ux43a-ux444ux438ux437ux438ux43aux430ux43bux438ux437ux43cux430}{%
\subsection{1.6. Логический тупик
физикализма}\label{ux43bux43eux433ux438ux447ux435ux441ux43aux438ux439-ux442ux443ux43fux438ux43a-ux444ux438ux437ux438ux43aux430ux43bux438ux437ux43cux430}}

Таким образом, физикализм сталкивается с совокупностью нерешаемых
проблем:

\begin{itemize}
\item
  материя принимается без объяснения;
\item
  сознание не выводится;
\item
  наблюдатель не имеет статуса;
\item
  смысл и опыт оказываются внешними теории.
\end{itemize}

Эти проблемы не являются временными. Они носят \textbf{онтологический
характер}.

Добавление новых физических теорий не устраняет тупик, поскольку он
заложен\\
в исходных допущениях.

\hypertarget{ux43dux435ux43eux431ux445ux43eux434ux438ux43cux43eux441ux442ux44c-ux441ux43cux435ux43dux44b-ux43eux43dux442ux43eux43bux43eux433ux438ux438}{%
\subsection{1.7. Необходимость смены
онтологии}\label{ux43dux435ux43eux431ux445ux43eux434ux438ux43cux43eux441ux442ux44c-ux441ux43cux435ux43dux44b-ux43eux43dux442ux43eux43bux43eux433ux438ux438}}

Если сознание:

\begin{itemize}
\item
  не выводимо из материи;
\item
  необходимо для описания реальности;
\item
  не может быть эпифеноменом,
\end{itemize}

то оно должно обладать фундаментальным онтологическим статусом.

Это не означает отказа от физики, но требует пересмотра оснований, на
которых она строится.

Физическая реальность должна рассматриваться как производная, а не как
первооснова.

\hypertarget{ux444ux438ux43aux441ux430ux446ux438ux44f-ux438ux441ux445ux43eux434ux43dux43eux439-ux43fux43eux437ux438ux446ux438ux438}{%
\subsection{1.8. Фиксация исходной
позиции}\label{ux444ux438ux43aux441ux430ux446ux438ux44f-ux438ux441ux445ux43eux434ux43dux43eux439-ux43fux43eux437ux438ux446ux438ux438}}

В рамках данной книги принимается следующая принципиальная фиксация:

\textbf{Сознание не может быть эпифеноменом.} Оно является
фундаментальной реальностью,\\
из которой выводятся материя, пространство, время и законы.

Эта фиксация не является произвольной. Она является логическим
следствием невозможности завершённого физикализма.

\hypertarget{ux43fux435ux440ux435ux445ux43eux434-ux43a-ux43dux43eux432ux43eux439-ux43eux43dux442ux43eux43bux43eux433ux438ux438}{%
\subsection{1.9. Переход к новой
онтологии}\label{ux43fux435ux440ux435ux445ux43eux434-ux43a-ux43dux43eux432ux43eux439-ux43eux43dux442ux43eux43bux43eux433ux438ux438}}

Отказ от физикализма не означает отказа от строгости.

Напротив, он открывает возможность построения онтологии, в которой:

\begin{itemize}
\item
  наблюдатель имеет статус;
\item
  знание становится формой бытия;
\item
  реальность описывается как процесс, а не как данность.
\end{itemize}

Именно этой онтологии посвящены следующие главы.

\end{document}
