% Options for packages loaded elsewhere
\PassOptionsToPackage{unicode}{hyperref}
\PassOptionsToPackage{hyphens}{url}
%
\documentclass[
]{article}
\usepackage{amsmath,amssymb}
\usepackage{lmodern}
\usepackage{iftex}
\ifPDFTeX
  \usepackage[T1]{fontenc}
  \usepackage[utf8]{inputenc}
  \usepackage{textcomp} % provide euro and other symbols
\else % if luatex or xetex
  \usepackage{unicode-math}
  \defaultfontfeatures{Scale=MatchLowercase}
  \defaultfontfeatures[\rmfamily]{Ligatures=TeX,Scale=1}
\fi
% Use upquote if available, for straight quotes in verbatim environments
\IfFileExists{upquote.sty}{\usepackage{upquote}}{}
\IfFileExists{microtype.sty}{% use microtype if available
  \usepackage[]{microtype}
  \UseMicrotypeSet[protrusion]{basicmath} % disable protrusion for tt fonts
}{}
\makeatletter
\@ifundefined{KOMAClassName}{% if non-KOMA class
  \IfFileExists{parskip.sty}{%
    \usepackage{parskip}
  }{% else
    \setlength{\parindent}{0pt}
    \setlength{\parskip}{6pt plus 2pt minus 1pt}}
}{% if KOMA class
  \KOMAoptions{parskip=half}}
\makeatother
\usepackage{xcolor}
\setlength{\emergencystretch}{3em} % prevent overfull lines
\providecommand{\tightlist}{%
  \setlength{\itemsep}{0pt}\setlength{\parskip}{0pt}}
\setcounter{secnumdepth}{-\maxdimen} % remove section numbering
\ifLuaTeX
  \usepackage{selnolig}  % disable illegal ligatures
\fi
\IfFileExists{bookmark.sty}{\usepackage{bookmark}}{\usepackage{hyperref}}
\IfFileExists{xurl.sty}{\usepackage{xurl}}{} % add URL line breaks if available
\urlstyle{same} % disable monospaced font for URLs
\hypersetup{
  hidelinks,
  pdfcreator={LaTeX via pandoc}}

\author{}
\date{}

\begin{document}

\hypertarget{ux433ux43bux430ux432ux430-14.-ux438ux442ux43eux433ux438-ux43eux43dux442ux43eux43bux43eux433ux438ux438-ux441ux43eux437ux43dux430ux43dux438ux44f}{%
\section{Глава 14. Итоги онтологии
Сознания}\label{ux433ux43bux430ux432ux430-14.-ux438ux442ux43eux433ux438-ux43eux43dux442ux43eux43bux43eux433ux438ux438-ux441ux43eux437ux43dux430ux43dux438ux44f}}

\hypertarget{ux446ux435ux43bux44c-ux438-ux441ux442ux430ux442ux443ux441-ux43eux43dux442ux43eux43bux43eux433ux438ux438}{%
\subsection{14.1. Цель и статус
онтологии}\label{ux446ux435ux43bux44c-ux438-ux441ux442ux430ux442ux443ux441-ux43eux43dux442ux43eux43bux43eux433ux438ux438}}

Целью первого тома данной книги было построение фундаментальной
онтологии, в которой Сознание рассматривается как первичная реальность,
а все формы бытия --- как производные структуры знания.

Данная онтология не является метафизической гипотезой, мировоззренческой
доктриной или философской интерпретацией физики.

Она представляет собой строго согласованную систему понятий,
обеспечивающую основание для формального описания реальности.

\hypertarget{ux43eux441ux43dux43eux432ux43dux44bux435-ux43eux43dux442ux43eux43bux43eux433ux438ux447ux435ux441ux43aux438ux435-ux43fux43eux43bux43eux436ux435ux43dux438ux44f}{%
\subsection{14.2. Основные онтологические
положения}\label{ux43eux441ux43dux43eux432ux43dux44bux435-ux43eux43dux442ux43eux43bux43eux433ux438ux447ux435ux441ux43aux438ux435-ux43fux43eux43bux43eux436ux435ux43dux438ux44f}}

В ходе изложения были последовательно установлены\\
следующие фундаментальные положения:

\begin{enumerate}
\def\labelenumi{\arabic{enumi}.}
\item
  \textbf{Сознание является фундаментальной реальностью}, не выводимой
  из материи, информации или физических законов.
\item
  \textbf{Пустота} представляет собой не отсутствие, а потенциал
  различий, из которого возникают все формы.
\item
  \textbf{Самодифференциация Сознания} является источником структуры и
  многообразия.
\item
  \textbf{Знание} есть форма бытия, а не отражение внешнего мира.
\item
  \textbf{Различие} является первоосновой пространства, времени и
  структуры.
\end{enumerate}

\hypertarget{ux43fux440ux43eux438ux437ux432ux43eux434ux43dux44bux435-ux441ux442ux440ux443ux43aux442ux443ux440ux44b-ux440ux435ux430ux43bux44cux43dux43eux441ux442ux438}{%
\subsection{14.3. Производные структуры
реальности}\label{ux43fux440ux43eux438ux437ux432ux43eux434ux43dux44bux435-ux441ux442ux440ux443ux43aux442ux443ux440ux44b-ux440ux435ux430ux43bux44cux43dux43eux441ux442ux438}}

На основе указанных принципов были выведены ключевые производные
понятия:

\begin{itemize}
\item
  пространство как структура различий;
\item
  время как процесс изменения различий;
\item
  наблюдатель как локализация знания;
\item
  материя как устойчивая конфигурация знания;
\item
  энергия как динамика изменений;
\item
  причинность как статистическая устойчивость.
\end{itemize}

Все эти понятия не вводятся аксиоматически, а возникают как следствия
онтологической структуры Сознания.

\hypertarget{ux441ux432ux43eux431ux43eux434ux430-ux441ux43cux44bux441ux43b-ux438-ux440ux430ux437ux432ux438ux442ux438ux435}{%
\subsection{14.4. Свобода, смысл и
развитие}\label{ux441ux432ux43eux431ux43eux434ux430-ux441ux43cux44bux441ux43b-ux438-ux440ux430ux437ux432ux438ux442ux438ux435}}

Было показано, что:

\begin{itemize}
\item
  неопределённость является онтологической;
\item
  будущее принципиально незамкнуто;
\item
  свобода есть недоопределённость будущего;
\item
  смысл является структурой знания;
\item
  ценность отражает устойчивость смысла;
\item
  развитие имеет направленность без цели.
\end{itemize}

Эти аспекты не противоречат строгости онтологии, а логически следуют из
неё.

\hypertarget{ux441ux430ux43cux43eux441ux43eux433ux43bux430ux441ux43eux432ux430ux43dux43dux43eux441ux442ux44c-ux43eux43dux442ux43eux43bux43eux433ux438ux438}{%
\subsection{14.5. Самосогласованность
онтологии}\label{ux441ux430ux43cux43eux441ux43eux433ux43bux430ux441ux43eux432ux430ux43dux43dux43eux441ux442ux44c-ux43eux43dux442ux43eux43bux43eux433ux438ux438}}

Онтология Сознания обладает следующими свойствами:

\begin{itemize}
\item
  внутренней логической замкнутостью;
\item
  отсутствием внешних сущностей;
\item
  непротиворечивостью понятий;
\item
  совместимостью с формализацией.
\end{itemize}

Каждое введённое понятие имеет строгое место в общей структуре теории.

\hypertarget{ux441ux432ux44fux437ux44c-ux441-ux444ux43eux440ux43cux430ux43bux438ux437ux43cux43eux43c-ux442ux43eux43cux430-ii}{%
\subsection{14.6. Связь с формализмом Тома
II}\label{ux441ux432ux44fux437ux44c-ux441-ux444ux43eux440ux43cux430ux43bux438ux437ux43cux43eux43c-ux442ux43eux43cux430-ii}}

Онтология, изложенная в первом томе, не является самостоятельной
философской конструкцией.

Она непосредственно реализуется в математическом формализме второго
тома, где:

\begin{itemize}
\item
  Сознание формализуется как пространство состояний;
\item
  различия --- как информационные меры;
\item
  динамика --- как геометрия изменений;
\item
  наблюдение --- как проекция;
\item
  гравитация --- как геометрия плотности знания.
\end{itemize}

Это подтверждает фундаментальный статус онтологии.

\hypertarget{ux43dux43eux432ux438ux437ux43dux430-ux438-ux43dux430ux443ux447ux43dux44bux439-ux441ux442ux430ux442ux443ux441}{%
\subsection{14.7. Новизна и научный
статус}\label{ux43dux43eux432ux438ux437ux43dux430-ux438-ux43dux430ux443ux447ux43dux44bux439-ux441ux442ux430ux442ux443ux441}}

Новизна онтологии Сознания заключается в том, что она:

\begin{itemize}
\item
  устраняет дуализм субъекта и объекта;
\item
  делает смысл онтологической категорией;
\item
  объясняет физическую реальность как вторичную;
\item
  объединяет онтологию и математику.
\end{itemize}

Онтология не подменяет физику, но предоставляет ей фундаментальное
основание, в котором физические законы являются частными проявлениями
более глубокой структуры.

\hypertarget{ux433ux440ux430ux43dux438ux446ux44b-ux438-ux43eux442ux43aux440ux44bux442ux43eux441ux442ux44c}{%
\subsection{14.8. Границы и
открытость}\label{ux433ux440ux430ux43dux438ux446ux44b-ux438-ux43eux442ux43aux440ux44bux442ux43eux441ux442ux44c}}

Онтология Сознания не претендует на окончательность.

Она открыта для:

\begin{itemize}
\item
  уточнения понятий;
\item
  расширения формализма;
\item
  применения к новым областям;
\item
  проверки на внутреннюю непротиворечивость.
\end{itemize}

Открытость не является слабостью, а отражает онтологическую
незамкнутость самой реальности.

\hypertarget{ux437ux430ux43aux43bux44eux447ux435ux43dux438ux435}{%
\subsection{14.9.
Заключение}\label{ux437ux430ux43aux43bux44eux447ux435ux43dux438ux435}}

В первом томе была построена цельная онтология Сознания, в которой:

\begin{itemize}
\item
  Сознание является основанием бытия;
\item
  различие --- источником структуры;
\item
  знание --- формой существования;
\item
  реальность --- процессом саморазворачивания Сознания.
\end{itemize}

Это завершает онтологическую часть работы и подготавливает переход к
строгой математической формализации, представленной во втором томе.

\end{document}
