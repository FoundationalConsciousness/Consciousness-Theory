% Options for packages loaded elsewhere
\PassOptionsToPackage{unicode}{hyperref}
\PassOptionsToPackage{hyphens}{url}
%
\documentclass[
]{article}
\usepackage{amsmath,amssymb}
\usepackage{lmodern}
\usepackage{iftex}
\ifPDFTeX
  \usepackage[T1]{fontenc}
  \usepackage[utf8]{inputenc}
  \usepackage{textcomp} % provide euro and other symbols
\else % if luatex or xetex
  \usepackage{unicode-math}
  \defaultfontfeatures{Scale=MatchLowercase}
  \defaultfontfeatures[\rmfamily]{Ligatures=TeX,Scale=1}
\fi
% Use upquote if available, for straight quotes in verbatim environments
\IfFileExists{upquote.sty}{\usepackage{upquote}}{}
\IfFileExists{microtype.sty}{% use microtype if available
  \usepackage[]{microtype}
  \UseMicrotypeSet[protrusion]{basicmath} % disable protrusion for tt fonts
}{}
\makeatletter
\@ifundefined{KOMAClassName}{% if non-KOMA class
  \IfFileExists{parskip.sty}{%
    \usepackage{parskip}
  }{% else
    \setlength{\parindent}{0pt}
    \setlength{\parskip}{6pt plus 2pt minus 1pt}}
}{% if KOMA class
  \KOMAoptions{parskip=half}}
\makeatother
\usepackage{xcolor}
\setlength{\emergencystretch}{3em} % prevent overfull lines
\providecommand{\tightlist}{%
  \setlength{\itemsep}{0pt}\setlength{\parskip}{0pt}}
\setcounter{secnumdepth}{-\maxdimen} % remove section numbering
\ifLuaTeX
  \usepackage{selnolig}  % disable illegal ligatures
\fi
\IfFileExists{bookmark.sty}{\usepackage{bookmark}}{\usepackage{hyperref}}
\IfFileExists{xurl.sty}{\usepackage{xurl}}{} % add URL line breaks if available
\urlstyle{same} % disable monospaced font for URLs
\hypersetup{
  hidelinks,
  pdfcreator={LaTeX via pandoc}}

\author{}
\date{}

\begin{document}

\hypertarget{ux433ux43bux430ux432ux430-9.-ux432ux440ux435ux43cux44f-ux43aux430ux43a-ux43fux440ux43eux446ux435ux441ux441-ux438ux437ux43cux435ux43dux435ux43dux438ux44f-ux440ux430ux437ux43bux438ux447ux438ux439}{%
\section{Глава 9. Время как процесс изменения
различий}\label{ux433ux43bux430ux432ux430-9.-ux432ux440ux435ux43cux44f-ux43aux430ux43a-ux43fux440ux43eux446ux435ux441ux441-ux438ux437ux43cux435ux43dux435ux43dux438ux44f-ux440ux430ux437ux43bux438ux447ux438ux439}}

\hypertarget{section}{%
\subsubsection{}\label{section}}

\hypertarget{ux43fux440ux43eux431ux43bux435ux43cux430-ux432ux440ux435ux43cux435ux43dux438-ux43aux430ux43a-ux43fux435ux440ux432ux438ux447ux43dux43eux439-ux441ux443ux449ux43dux43eux441ux442ux438}{%
\subsection{9.1. Проблема времени как первичной
сущности}\label{ux43fux440ux43eux431ux43bux435ux43cux430-ux432ux440ux435ux43cux435ux43dux438-ux43aux430ux43a-ux43fux435ux440ux432ux438ux447ux43dux43eux439-ux441ux443ux449ux43dux43eux441ux442ux438}}

Во многих философских и физических теориях время рассматривается как
фундаментальный параметр, в котором разворачиваются события и процессы.
Однако такой подход оставляет без ответа принципиальный вопрос:
\textbf{что именно делает время временем}?

Если время первично, то:

\begin{itemize}
\item
  почему оно обладает направленностью;
\item
  откуда возникает различие между прошлым и будущим;
\item
  каким образом время связано с изменениями, а не просто с их
  параметризацией?
\end{itemize}

В рамках онтологии Сознания время не может быть первоосновой, поскольку
само предполагает наличие изменений и различий.

\hypertarget{ux432ux440ux435ux43cux44f-ux43dux435-ux44fux432ux43bux44fux435ux442ux441ux44f-ux43aux43eux43dux442ux435ux439ux43dux435ux440ux43eux43c-ux441ux43eux431ux44bux442ux438ux439}{%
\subsection{9.2. Время не является контейнером
событий}\label{ux432ux440ux435ux43cux44f-ux43dux435-ux44fux432ux43bux44fux435ux442ux441ux44f-ux43aux43eux43dux442ux435ux439ux43dux435ux440ux43eux43c-ux441ux43eux431ux44bux442ux438ux439}}

Подобно пространству, время часто мыслится как контейнер, в котором
«расположены» события.

Однако если отсутствуют различия, то отсутствуют и события, а значит,
отсутствует и время.

В онтологической Пустоте невозможно говорить ни о прошлом, ни о
настоящем, ни о будущем.

Время возникает только тогда, когда появляется \textbf{изменение
различий}.

\hypertarget{ux43eux43dux442ux43eux43bux43eux433ux438ux447ux435ux441ux43aux43eux435-ux43eux43fux440ux435ux434ux435ux43bux435ux43dux438ux435-ux432ux440ux435ux43cux435ux43dux438}{%
\subsection{9.3. Онтологическое определение
времени}\label{ux43eux43dux442ux43eux43bux43eux433ux438ux447ux435ux441ux43aux43eux435-ux43eux43fux440ux435ux434ux435ux43bux435ux43dux438ux435-ux432ux440ux435ux43cux435ux43dux438}}

\textbf{Время} определяется как \textbf{процесс изменения различий в
Сознании}.

Это определение принципиально:

\begin{itemize}
\item
  не сводит время к параметру;
\item
  не предполагает внешнего течения;
\item
  не требует абсолютной шкалы.
\end{itemize}

Время есть не то, \emph{в чём} происходят изменения, а само
\textbf{изменение как таковое}, рассматриваемое в онтологическом смысле.

\hypertarget{ux43fux43eux441ux43bux435ux434ux43eux432ux430ux442ux435ux43bux44cux43dux43eux441ux442ux44c-ux438-ux43dux430ux43fux440ux430ux432ux43bux435ux43dux43dux43eux441ux442ux44c}{%
\subsection{9.4. Последовательность и
направленность}\label{ux43fux43eux441ux43bux435ux434ux43eux432ux430ux442ux435ux43bux44cux43dux43eux441ux442ux44c-ux438-ux43dux430ux43fux440ux430ux432ux43bux435ux43dux43dux43eux441ux442ux44c}}

Изменение различий порождает \textbf{последовательность}.

Когда формы знания сменяют друг друга, возникает порядок, в котором:

\begin{itemize}
\item
  одни конфигурации различий предшествуют другим;
\item
  новые формы зависят от уже возникших.
\end{itemize}

Направленность времени (стрела времени) является следствием асимметрии
процессов самодифференциации и стабилизации форм знания.

\hypertarget{ux43dux435ux43eux431ux440ux430ux442ux438ux43cux43eux441ux442ux44c-ux438-ux443ux441ux442ux43eux439ux447ux438ux432ux43eux441ux442ux44c}{%
\subsection{9.5. Необратимость и
устойчивость}\label{ux43dux435ux43eux431ux440ux430ux442ux438ux43cux43eux441ux442ux44c-ux438-ux443ux441ux442ux43eux439ux447ux438ux432ux43eux441ux442ux44c}}

Необратимость времени связана с возникновением устойчивых форм знания.

Когда форма знания стабилизируется, она ограничивает возможные будущие
различия, но не может быть полностью «стерта» без нарушения всей
структуры.

Таким образом, прошлое сохраняется в виде устойчивых форм, а будущее
остаётся открытым в пределах неопределённости.

\hypertarget{ux432ux440ux435ux43cux44f-ux438-ux43dux435ux43eux43fux440ux435ux434ux435ux43bux451ux43dux43dux43eux441ux442ux44c}{%
\subsection{9.6. Время и
неопределённость}\label{ux432ux440ux435ux43cux44f-ux438-ux43dux435ux43eux43fux440ux435ux434ux435ux43bux451ux43dux43dux43eux441ux442ux44c}}

Время не устраняет неопределённость, а перераспределяет её.

Каждый акт изменения различий:

\begin{itemize}
\item
  уменьшает неопределённость в одном аспекте;
\item
  увеличивает её в другом.
\end{itemize}

Эта динамика является онтологическим основанием вероятностного описания
процессов, которое будет формализовано в Томе II.

\hypertarget{ux43bux43eux43aux430ux43bux44cux43dux43eux435-ux438-ux433ux43bux43eux431ux430ux43bux44cux43dux43eux435-ux432ux440ux435ux43cux44f}{%
\subsection{9.7. Локальное и глобальное
время}\label{ux43bux43eux43aux430ux43bux44cux43dux43eux435-ux438-ux433ux43bux43eux431ux430ux43bux44cux43dux43eux435-ux432ux440ux435ux43cux44f}}

Поскольку время возникает как процесс изменения различий, оно не обязано
быть универсальным и единым.

Различные структуры различий могут обладать различной динамикой, что
приводит к существованию \textbf{локальных времён}.

Глобальное время является абстракцией, возникающей при рассмотрении
устойчивых и согласованных структур различий.

\hypertarget{ux432ux440ux435ux43cux44f-ux438-ux43dux430ux431ux43bux44eux434ux430ux442ux435ux43bux44c}{%
\subsection{9.8. Время и
наблюдатель}\label{ux432ux440ux435ux43cux44f-ux438-ux43dux430ux431ux43bux44eux434ux430ux442ux435ux43bux44c}}

Наблюдатель фиксирует изменения различий, тем самым локализуя время.

Время наблюдения --- это не объективный поток, а последовательность
актуализированных форм знания.

Это положение позволяет естественным образом объяснить релятивистские
эффекты без введения внешних постулатов.

\hypertarget{ux43fux440ux435ux434ux432ux43eux441ux445ux438ux449ux435ux43dux438ux435-ux444ux43eux440ux43cux430ux43bux438ux437ux430ux446ux438ux438}{%
\subsection{9.9. Предвосхищение
формализации}\label{ux43fux440ux435ux434ux432ux43eux441ux445ux438ux449ux435ux43dux438ux435-ux444ux43eux440ux43cux430ux43bux438ux437ux430ux446ux438ux438}}

В Томе II время будет представлено как параметр динамики состояний в
пространстве различий.

Однако важно подчеркнуть, что параметризация времени является
\textbf{вторичной} по отношению к онтологическому процессу изменения
различий.

Формулы описывают время, но не создают его.

\hypertarget{ux437ux43dux430ux447ux435ux43dux438ux435-ux433ux43bux430ux432ux44b}{%
\subsection{9.10. Значение
главы}\label{ux437ux43dux430ux447ux435ux43dux438ux435-ux433ux43bux430ux432ux44b}}

В этой главе было показано, что:

\begin{itemize}
\item
  время не является фундаментальной сущностью;
\item
  оно возникает как процесс изменения различий;
\item
  направленность и необратимость времени имеют онтологическое
  объяснение;
\item
  время связано с неопределённостью и устойчивостью форм знания.
\end{itemize}

Тем самым завершено построение базовой онтологии пространства и времени
в рамках Теории Сознания.

В следующей главе будет рассмотрен \textbf{наблюдатель как локализация
знания}, что позволит связать онтологию с физическими теориями и
эмпирическими наблюдениями.

\end{document}
