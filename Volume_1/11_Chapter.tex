% Options for packages loaded elsewhere
\PassOptionsToPackage{unicode}{hyperref}
\PassOptionsToPackage{hyphens}{url}
%
\documentclass[
]{article}
\usepackage{amsmath,amssymb}
\usepackage{lmodern}
\usepackage{iftex}
\ifPDFTeX
  \usepackage[T1]{fontenc}
  \usepackage[utf8]{inputenc}
  \usepackage{textcomp} % provide euro and other symbols
\else % if luatex or xetex
  \usepackage{unicode-math}
  \defaultfontfeatures{Scale=MatchLowercase}
  \defaultfontfeatures[\rmfamily]{Ligatures=TeX,Scale=1}
\fi
% Use upquote if available, for straight quotes in verbatim environments
\IfFileExists{upquote.sty}{\usepackage{upquote}}{}
\IfFileExists{microtype.sty}{% use microtype if available
  \usepackage[]{microtype}
  \UseMicrotypeSet[protrusion]{basicmath} % disable protrusion for tt fonts
}{}
\makeatletter
\@ifundefined{KOMAClassName}{% if non-KOMA class
  \IfFileExists{parskip.sty}{%
    \usepackage{parskip}
  }{% else
    \setlength{\parindent}{0pt}
    \setlength{\parskip}{6pt plus 2pt minus 1pt}}
}{% if KOMA class
  \KOMAoptions{parskip=half}}
\makeatother
\usepackage{xcolor}
\setlength{\emergencystretch}{3em} % prevent overfull lines
\providecommand{\tightlist}{%
  \setlength{\itemsep}{0pt}\setlength{\parskip}{0pt}}
\setcounter{secnumdepth}{-\maxdimen} % remove section numbering
\ifLuaTeX
  \usepackage{selnolig}  % disable illegal ligatures
\fi
\IfFileExists{bookmark.sty}{\usepackage{bookmark}}{\usepackage{hyperref}}
\IfFileExists{xurl.sty}{\usepackage{xurl}}{} % add URL line breaks if available
\urlstyle{same} % disable monospaced font for URLs
\hypersetup{
  hidelinks,
  pdfcreator={LaTeX via pandoc}}

\author{}
\date{}

\begin{document}

\hypertarget{ux433ux43bux430ux432ux430-11.-ux43cux430ux442ux435ux440ux438ux44f-ux44dux43dux435ux440ux433ux438ux44f-ux438-ux43fux440ux438ux447ux438ux43dux43dux43eux441ux442ux44c}{%
\section{Глава 11. Материя, энергия и
причинность}\label{ux433ux43bux430ux432ux430-11.-ux43cux430ux442ux435ux440ux438ux44f-ux44dux43dux435ux440ux433ux438ux44f-ux438-ux43fux440ux438ux447ux438ux43dux43dux43eux441ux442ux44c}}

\hypertarget{section}{%
\subsubsection{}\label{section}}

\hypertarget{ux43fux440ux43eux431ux43bux435ux43cux430-ux444ux438ux437ux438ux447ux435ux441ux43aux43eux439-ux440ux435ux430ux43bux44cux43dux43eux441ux442ux438}{%
\subsection{11.1. Проблема физической
реальности}\label{ux43fux440ux43eux431ux43bux435ux43cux430-ux444ux438ux437ux438ux447ux435ux441ux43aux43eux439-ux440ux435ux430ux43bux44cux43dux43eux441ux442ux438}}

Классическая физика исходит из предположения, что материя и энергия
являются фундаментальными сущностями, а причинность --- базовым
принципом устройства мира.

Однако в рамках онтологии Сознания возникает необходимость переосмыслить
эти понятия:

\begin{itemize}
\item
  материя не может быть первичной, если первична Пустота и различие;
\item
  энергия не может быть сущностью, если изменения являются процессами;
\item
  причинность не может быть абсолютной, если знание локализовано.
\end{itemize}

Задача этой главы --- показать, что материя, энергия и причинность
являются \textbf{производными онтологическими структурами}, а не
фундаментальными основаниями реальности.

\hypertarget{ux43cux430ux442ux435ux440ux438ux44f-ux43aux430ux43a-ux443ux441ux442ux43eux439ux447ux438ux432ux430ux44f-ux43aux43eux43dux444ux438ux433ux443ux440ux430ux446ux438ux44f-ux437ux43dux430ux43dux438ux44f}{%
\subsection{11.2. Материя как устойчивая конфигурация
знания}\label{ux43cux430ux442ux435ux440ux438ux44f-ux43aux430ux43a-ux443ux441ux442ux43eux439ux447ux438ux432ux430ux44f-ux43aux43eux43dux444ux438ux433ux443ux440ux430ux446ux438ux44f-ux437ux43dux430ux43dux438ux44f}}

\textbf{Материя} определяется как \textbf{устойчивая конфигурация форм
знания}, поддерживаемая в процессе самодифференциации Сознания.

Материальные объекты:

\begin{itemize}
\item
  не существуют как независимые «вещи»;
\item
  представляют собой стабилизированные структуры различий;
\item
  сохраняют свою идентичность благодаря устойчивости этих различий во
  времени.
\end{itemize}

То, что воспринимается как «объект», есть устойчивый паттерн знания,
воспроизводимый при различных локализациях наблюдателя.

\hypertarget{ux43eux431ux44aux435ux43aux442ux438ux432ux43dux43eux441ux442ux44c-ux43cux430ux442ux435ux440ux438ux438}{%
\subsection{11.3. Объективность
материи}\label{ux43eux431ux44aux435ux43aux442ux438ux432ux43dux43eux441ux442ux44c-ux43cux430ux442ux435ux440ux438ux438}}

Объективность материи не означает её независимость от Сознания.

Она означает:

\begin{itemize}
\item
  высокую степень устойчивости форм знания;
\item
  согласованность между различными наблюдателями;
\item
  воспроизводимость структур различий.
\end{itemize}

Материя объективна не потому, что она «вне» Сознания, а потому, что она
\textbf{стабильна внутри него}.

\hypertarget{ux44dux43dux435ux440ux433ux438ux44f-ux43aux430ux43a-ux434ux438ux43dux430ux43cux438ux43aux430-ux438ux437ux43cux435ux43dux435ux43dux438ux439}{%
\subsection{11.4. Энергия как динамика
изменений}\label{ux44dux43dux435ux440ux433ux438ux44f-ux43aux430ux43a-ux434ux438ux43dux430ux43cux438ux43aux430-ux438ux437ux43cux435ux43dux435ux43dux438ux439}}

\textbf{Энергия} не является субстанцией.

Энергия определяется как \textbf{мера интенсивности процессов изменения
различий}.

Когда формы знания изменяются:

\begin{itemize}
\item
  быстро или медленно;
\item
  локально или глобально;
\item
  согласованно или хаотично,
\end{itemize}

эти изменения могут быть количественно охарактеризованы --- именно это и
выражается понятием энергии.

\hypertarget{ux441ux43eux445ux440ux430ux43dux435ux43dux438ux435-ux44dux43dux435ux440ux433ux438ux438}{%
\subsection{11.5. Сохранение
энергии}\label{ux441ux43eux445ux440ux430ux43dux435ux43dux438ux435-ux44dux43dux435ux440ux433ux438ux438}}

Закон сохранения энергии отражает не сохранение «вещества», а
\textbf{устойчивость общей динамики изменений} в замкнутых структурах
различий.

Если структура замкнута относительно внешних различий, суммарная
интенсивность изменений остаётся постоянной.

Таким образом, сохранение энергии является следствием структурной
устойчивости, а не фундаментальной аксиомой.

\hypertarget{ux43fux440ux438ux447ux438ux43dux43dux43eux441ux442ux44c-ux43aux430ux43a-ux441ux442ux430ux442ux438ux441ux442ux438ux447ux435ux441ux43aux430ux44f-ux443ux441ux442ux43eux439ux447ux438ux432ux43eux441ux442ux44c}{%
\subsection{11.6. Причинность как статистическая
устойчивость}\label{ux43fux440ux438ux447ux438ux43dux43dux43eux441ux442ux44c-ux43aux430ux43a-ux441ux442ux430ux442ux438ux441ux442ux438ux447ux435ux441ux43aux430ux44f-ux443ux441ux442ux43eux439ux447ux438ux432ux43eux441ux442ux44c}}

В классическом понимании причинность предполагает жёсткую детерминацию:
каждое событие имеет однозначную причину.

В рамках Теории Сознания \textbf{причинность определяется как
статистически устойчивая связь между изменениями различий}.

Причинные связи:

\begin{itemize}
\item
  не являются абсолютными;
\item
  возникают из повторяемости процессов;
\item
  зависят от масштаба и уровня описания.
\end{itemize}

\hypertarget{ux432ux435ux440ux43eux44fux442ux43dux43eux441ux442ux44c-ux438-ux43fux440ux438ux447ux438ux43dux43dux43eux441ux442ux44c}{%
\subsection{11.7. Вероятность и
причинность}\label{ux432ux435ux440ux43eux44fux442ux43dux43eux441ux442ux44c-ux438-ux43fux440ux438ux447ux438ux43dux43dux43eux441ux442ux44c}}

На фундаментальном уровне изменения различий не детерминированы жёстко.

Однако устойчивые структуры создают высоковероятные траектории развития,
которые и интерпретируются как причинные цепочки.

Таким образом:

\begin{itemize}
\item
  вероятность является фундаментальной;
\item
  причинность --- её макроскопическим проявлением.
\end{itemize}

\hypertarget{ux43cux430ux442ux435ux440ux438ux44f-ux44dux43dux435ux440ux433ux438ux44f-ux438-ux432ux440ux435ux43cux44f}{%
\subsection{11.8. Материя, энергия и
время}\label{ux43cux430ux442ux435ux440ux438ux44f-ux44dux43dux435ux440ux433ux438ux44f-ux438-ux432ux440ux435ux43cux44f}}

Материя существует благодаря устойчивости во времени.

Энергия проявляется как изменение во времени.

Причинность связывает изменения во времени в устойчивые
последовательности.

Все три понятия оказываются неразрывно связанными с временем как
процессом изменения различий.

\hypertarget{ux43fux440ux435ux434ux432ux43eux441ux445ux438ux449ux435ux43dux438ux435-ux444ux438ux437ux438ux447ux435ux441ux43aux43eux439-ux444ux43eux440ux43cux430ux43bux438ux437ux430ux446ux438ux438}{%
\subsection{11.9. Предвосхищение физической
формализации}\label{ux43fux440ux435ux434ux432ux43eux441ux445ux438ux449ux435ux43dux438ux435-ux444ux438ux437ux438ux447ux435ux441ux43aux43eux439-ux444ux43eux440ux43cux430ux43bux438ux437ux430ux446ux438ux438}}

В Томе II:

\begin{itemize}
\item
  материя будет описана как устойчивые состояния в пространстве
  состояний;
\item
  энергия --- как функционал динамики;
\item
  причинность --- как статистическая структура переходов.
\end{itemize}

Физические уравнения окажутся формальным выражением этих онтологических
принципов.

\hypertarget{ux437ux43dux430ux447ux435ux43dux438ux435-ux433ux43bux430ux432ux44b}{%
\subsection{11.10. Значение
главы}\label{ux437ux43dux430ux447ux435ux43dux438ux435-ux433ux43bux430ux432ux44b}}

В этой главе было показано, что:

\begin{itemize}
\item
  материя является устойчивой формой знания;
\item
  энергия --- динамикой изменений различий;
\item
  причинность --- статистической устойчивостью процессов.
\end{itemize}

Тем самым физическая реальность оказывается не внешней по отношению к
Сознанию, а \textbf{его устойчивым проявлением}.

\end{document}
