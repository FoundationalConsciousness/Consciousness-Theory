% Options for packages loaded elsewhere
\PassOptionsToPackage{unicode}{hyperref}
\PassOptionsToPackage{hyphens}{url}
%
\documentclass[
]{article}
\usepackage{amsmath,amssymb}
\usepackage{lmodern}
\usepackage{iftex}
\ifPDFTeX
  \usepackage[T1]{fontenc}
  \usepackage[utf8]{inputenc}
  \usepackage{textcomp} % provide euro and other symbols
\else % if luatex or xetex
  \usepackage{unicode-math}
  \defaultfontfeatures{Scale=MatchLowercase}
  \defaultfontfeatures[\rmfamily]{Ligatures=TeX,Scale=1}
\fi
% Use upquote if available, for straight quotes in verbatim environments
\IfFileExists{upquote.sty}{\usepackage{upquote}}{}
\IfFileExists{microtype.sty}{% use microtype if available
  \usepackage[]{microtype}
  \UseMicrotypeSet[protrusion]{basicmath} % disable protrusion for tt fonts
}{}
\makeatletter
\@ifundefined{KOMAClassName}{% if non-KOMA class
  \IfFileExists{parskip.sty}{%
    \usepackage{parskip}
  }{% else
    \setlength{\parindent}{0pt}
    \setlength{\parskip}{6pt plus 2pt minus 1pt}}
}{% if KOMA class
  \KOMAoptions{parskip=half}}
\makeatother
\usepackage{xcolor}
\setlength{\emergencystretch}{3em} % prevent overfull lines
\providecommand{\tightlist}{%
  \setlength{\itemsep}{0pt}\setlength{\parskip}{0pt}}
\setcounter{secnumdepth}{-\maxdimen} % remove section numbering
\ifLuaTeX
  \usepackage{selnolig}  % disable illegal ligatures
\fi
\IfFileExists{bookmark.sty}{\usepackage{bookmark}}{\usepackage{hyperref}}
\IfFileExists{xurl.sty}{\usepackage{xurl}}{} % add URL line breaks if available
\urlstyle{same} % disable monospaced font for URLs
\hypersetup{
  hidelinks,
  pdfcreator={LaTeX via pandoc}}

\author{}
\date{}

\begin{document}

\hypertarget{ux433ux43bux430ux432ux430-8.-ux43fux440ux43eux441ux442ux440ux430ux43dux441ux442ux432ux43e-ux43aux430ux43a-ux441ux442ux440ux443ux43aux442ux443ux440ux430-ux440ux430ux437ux43bux438ux447ux438ux439}{%
\section{Глава 8. Пространство как структура
различий}\label{ux433ux43bux430ux432ux430-8.-ux43fux440ux43eux441ux442ux440ux430ux43dux441ux442ux432ux43e-ux43aux430ux43a-ux441ux442ux440ux443ux43aux442ux443ux440ux430-ux440ux430ux437ux43bux438ux447ux438ux439}}

\hypertarget{section}{%
\subsubsection{}\label{section}}

\hypertarget{ux43fux440ux43eux431ux43bux435ux43cux430-ux43fux440ux43eux441ux442ux440ux430ux43dux441ux442ux432ux430-ux43aux430ux43a-ux43fux435ux440ux432ux438ux447ux43dux43eux439-ux441ux443ux449ux43dux43eux441ux442ux438}{%
\subsection{8.1. Проблема пространства как первичной
сущности}\label{ux43fux440ux43eux431ux43bux435ux43cux430-ux43fux440ux43eux441ux442ux440ux430ux43dux441ux442ux432ux430-ux43aux430ux43a-ux43fux435ux440ux432ux438ux447ux43dux43eux439-ux441ux443ux449ux43dux43eux441ux442ux438}}

В классических физических и философских онтологиях пространство
рассматривается как нечто первичное: либо как контейнер, в котором
располагаются объекты, либо как фундаментальная геометрическая
структура, заданная априори.

Однако такой подход сталкивается с рядом принципиальных трудностей:

\begin{itemize}
\item
  пространство предполагается существующим до объектов, но его структура
  остаётся необъяснённой;
\item
  метрика пространства вводится как данность;
\item
  связь между пространством и наблюдателем оказывается внешней.
\end{itemize}

В рамках онтологии Сознания пространство не может быть первоосновой,
поскольку само предполагает наличие различий и отношений.

\hypertarget{ux43fux440ux43eux441ux442ux440ux430ux43dux441ux442ux432ux43e-ux43dux435-ux44fux432ux43bux44fux435ux442ux441ux44f-ux43aux43eux43dux442ux435ux439ux43dux435ux440ux43eux43c}{%
\subsection{8.2. Пространство не является
контейнером}\label{ux43fux440ux43eux441ux442ux440ux430ux43dux441ux442ux432ux43e-ux43dux435-ux44fux432ux43bux44fux435ux442ux441ux44f-ux43aux43eux43dux442ux435ux439ux43dux435ux440ux43eux43c}}

Пространство не существует как пустой «вместилище», ожидающее наполнения
объектами.

Идея пространства-контейнера является следствием
объектно-ориентированной онтологии, где вещи считаются первичными.

В предлагаемой теории пространство возникает \textbf{из различий}, а не
наоборот.

Если отсутствуют различия, отсутствует и пространство. Говорить о
пространстве в онтологической Пустоте бессмысленно.

\hypertarget{ux43fux440ux43eux441ux442ux440ux430ux43dux441ux442ux432ux43e-ux43aux430ux43a-ux441ux442ux440ux443ux43aux442ux443ux440ux438ux440ux43eux432ux430ux43dux43dux43eux435-ux43cux43dux43eux436ux435ux441ux442ux432ux43e-ux440ux430ux437ux43bux438ux447ux438ux439}{%
\subsection{8.3. Пространство как структурированное множество
различий}\label{ux43fux440ux43eux441ux442ux440ux430ux43dux441ux442ux432ux43e-ux43aux430ux43a-ux441ux442ux440ux443ux43aux442ux443ux440ux438ux440ux43eux432ux430ux43dux43dux43eux435-ux43cux43dux43eux436ux435ux441ux442ux432ux43e-ux440ux430ux437ux43bux438ux447ux438ux439}}

\textbf{Пространство} определяется как \textbf{структурированное
множество различий}, допускающее введение меры близости.

Это определение принципиально:

\begin{itemize}
\item
  не предполагает физические координаты;
\item
  не требует предварительной геометрии;
\item
  не вводит абсолютной размерности.
\end{itemize}

Пространство возникает тогда, когда различия образуют устойчивую
структуру отношений, в которой можно говорить о «ближе» и «дальше» в
онтологическом смысле.

\hypertarget{ux432ux43eux437ux43dux438ux43aux43dux43eux432ux435ux43dux438ux435-ux43cux435ux442ux440ux438ux43aux438}{%
\subsection{8.4. Возникновение
метрики}\label{ux432ux43eux437ux43dux438ux43aux43dux43eux432ux435ux43dux438ux435-ux43cux435ux442ux440ux438ux43aux438}}

Метрика пространства не задаётся заранее. Она возникает как способ
количественного выражения различий между формами знания.

Когда различия становятся сравнимыми, появляется возможность:

\begin{itemize}
\item
  измерять степень различия;
\item
  упорядочивать формы знания;
\item
  вводить геометрические представления.
\end{itemize}

Именно на этом этапе становятся применимыми математические инструменты,
такие как расстояния и дивергенции, которые будут подробно рассмотрены в
Томе II.

\hypertarget{ux440ux430ux437ux43cux435ux440ux43dux43eux441ux442ux44c-ux43aux430ux43a-ux43fux440ux43eux438ux437ux432ux43eux434ux43dux430ux44f-ux445ux430ux440ux430ux43aux442ux435ux440ux438ux441ux442ux438ux43aux430}{%
\subsection{8.5. Размерность как производная
характеристика}\label{ux440ux430ux437ux43cux435ux440ux43dux43eux441ux442ux44c-ux43aux430ux43a-ux43fux440ux43eux438ux437ux432ux43eux434ux43dux430ux44f-ux445ux430ux440ux430ux43aux442ux435ux440ux438ux441ux442ux438ux43aux430}}

Размерность пространства не является фундаментальной.

Она определяется:

\begin{itemize}
\item
  сложностью структуры различий;
\item
  количеством независимых направлений изменения;
\item
  устойчивостью форм знания.
\end{itemize}

Различные уровни реальности могут обладать различной эффективной
размерностью, что естественным образом объясняет появление как
классических, так и квантовых пространств.

\hypertarget{ux43bux43eux43aux430ux43bux44cux43dux43eux441ux442ux44c-ux438-ux43dux435ux43bux43eux43aux430ux43bux44cux43dux43eux441ux442ux44c}{%
\subsection{8.6. Локальность и
нелокальность}\label{ux43bux43eux43aux430ux43bux44cux43dux43eux441ux442ux44c-ux438-ux43dux435ux43bux43eux43aux430ux43bux44cux43dux43eux441ux442ux44c}}

Локальность в физическом смысле возникает как следствие структуры
различий, а не как аксиома.

Если формы знания близки в пространстве различий, они проявляются как
локальные.\\
Если же различия малы независимо от физического расстояния, возникает
эффект нелокальности.

Таким образом, квантовая нелокальность получает онтологическое
объяснение без нарушения фундаментальных принципов.

\hypertarget{ux43fux440ux43eux441ux442ux440ux430ux43dux441ux442ux432ux43e-ux438-ux43dux430ux431ux43bux44eux434ux430ux442ux435ux43bux44c}{%
\subsection{8.7. Пространство и
наблюдатель}\label{ux43fux440ux43eux441ux442ux440ux430ux43dux441ux442ux432ux43e-ux438-ux43dux430ux431ux43bux44eux434ux430ux442ux435ux43bux44c}}

Наблюдатель не находится «внутри» пространства как внешний объект. Он
является частью структуры различий, формирующей пространство.

Локализация наблюдателя означает фиксацию определённого подмножества
форм знания, относительно которых определяется структура пространства.

Это положение устраняет парадокс внешнего наблюдателя и подготавливает
понимание относительности пространственных характеристик.

\hypertarget{ux43fux440ux43eux441ux442ux440ux430ux43dux441ux442ux432ux43e-ux438-ux443ux441ux442ux43eux439ux447ux438ux432ux43eux441ux442ux44c-ux444ux43eux440ux43c}{%
\subsection{8.8. Пространство и устойчивость
форм}\label{ux43fux440ux43eux441ux442ux440ux430ux43dux441ux442ux432ux43e-ux438-ux443ux441ux442ux43eux439ux447ux438ux432ux43eux441ux442ux44c-ux444ux43eux440ux43c}}

Устойчивые формы знания формируют устойчивые геометрические структуры.

Именно поэтому пространство воспринимается как стабильное и объективное.
Однако эта стабильность является следствием устойчивости форм знания, а
не самостоятельным свойством пространства.

При изменении структуры различий геометрия пространства также может
изменяться.

\hypertarget{ux43fux440ux435ux434ux432ux43eux441ux445ux438ux449ux435ux43dux438ux435-ux444ux43eux440ux43cux430ux43bux438ux437ux430ux446ux438ux438}{%
\subsection{8.9. Предвосхищение
формализации}\label{ux43fux440ux435ux434ux432ux43eux441ux445ux438ux449ux435ux43dux438ux435-ux444ux43eux440ux43cux430ux43bux438ux437ux430ux446ux438ux438}}

На данном этапе можно сделать ключевое утверждение:

Геометрия пространства является отражением структуры различий в
Сознании.

В Томе II это утверждение будет выражено формально через:

\begin{itemize}
\item
  пространства состояний;
\item
  гильбертовы пространства;
\item
  информационные метрики.
\end{itemize}

Однако важно подчеркнуть, что формализация не вводит пространство, а
лишь описывает уже существующую онтологическую структуру.

\hypertarget{ux437ux43dux430ux447ux435ux43dux438ux435-ux433ux43bux430ux432ux44b}{%
\subsection{8.10. Значение
главы}\label{ux437ux43dux430ux447ux435ux43dux438ux435-ux433ux43bux430ux432ux44b}}

В этой главе было показано, что:

\begin{itemize}
\item
  пространство не является первичной сущностью;
\item
  оно возникает как структура различий;
\item
  метрика и размерность являются производными;
\item
  локальность и нелокальность имеют онтологическое объяснение.
\end{itemize}

Тем самым создано основание для рассмотрения \textbf{времени как
процесса изменения различий}, что и будет сделано в следующей главе.

\end{document}
