% Options for packages loaded elsewhere
\PassOptionsToPackage{unicode}{hyperref}
\PassOptionsToPackage{hyphens}{url}
%
\documentclass[
]{article}
\usepackage{amsmath,amssymb}
\usepackage{lmodern}
\usepackage{iftex}
\ifPDFTeX
  \usepackage[T1]{fontenc}
  \usepackage[utf8]{inputenc}
  \usepackage{textcomp} % provide euro and other symbols
\else % if luatex or xetex
  \usepackage{unicode-math}
  \defaultfontfeatures{Scale=MatchLowercase}
  \defaultfontfeatures[\rmfamily]{Ligatures=TeX,Scale=1}
\fi
% Use upquote if available, for straight quotes in verbatim environments
\IfFileExists{upquote.sty}{\usepackage{upquote}}{}
\IfFileExists{microtype.sty}{% use microtype if available
  \usepackage[]{microtype}
  \UseMicrotypeSet[protrusion]{basicmath} % disable protrusion for tt fonts
}{}
\makeatletter
\@ifundefined{KOMAClassName}{% if non-KOMA class
  \IfFileExists{parskip.sty}{%
    \usepackage{parskip}
  }{% else
    \setlength{\parindent}{0pt}
    \setlength{\parskip}{6pt plus 2pt minus 1pt}}
}{% if KOMA class
  \KOMAoptions{parskip=half}}
\makeatother
\usepackage{xcolor}
\setlength{\emergencystretch}{3em} % prevent overfull lines
\providecommand{\tightlist}{%
  \setlength{\itemsep}{0pt}\setlength{\parskip}{0pt}}
\setcounter{secnumdepth}{-\maxdimen} % remove section numbering
\ifLuaTeX
  \usepackage{selnolig}  % disable illegal ligatures
\fi
\IfFileExists{bookmark.sty}{\usepackage{bookmark}}{\usepackage{hyperref}}
\IfFileExists{xurl.sty}{\usepackage{xurl}}{} % add URL line breaks if available
\urlstyle{same} % disable monospaced font for URLs
\hypersetup{
  hidelinks,
  pdfcreator={LaTeX via pandoc}}

\author{}
\date{}

\begin{document}

\hypertarget{ux43eux43dux442ux43eux43bux43eux433ux438ux447ux435ux441ux43aux438ux439-ux441ux43bux43eux432ux430ux440ux44c}{%
\section{ОНТОЛОГИЧЕСКИЙ
СЛОВАРЬ}\label{ux43eux43dux442ux43eux43bux43eux433ux438ux447ux435ux441ux43aux438ux439-ux441ux43bux43eux432ux430ux440ux44c}}

\hypertarget{ux43aux430ux43dux43eux43dux438ux447ux435ux441ux43aux430ux44f-ux432ux435ux440ux441ux438ux44f-v0.1}{%
\subsection{(Каноническая версия
v0.1)}\label{ux43aux430ux43dux43eux43dux438ux447ux435ux441ux43aux430ux44f-ux432ux435ux440ux441ux438ux44f-v0.1}}

\hypertarget{ux441ux43eux437ux43dux430ux43dux438ux435}{%
\subsubsection{1.
Сознание}\label{ux441ux43eux437ux43dux430ux43dux438ux435}}

\textbf{Сознание} --- фундаментальная реальность, не являющаяся
объектом, процессом или свойством, в пределах которой возникают все
формы знания, различия и структуры.

Сознание не выводится из материи, информации или физики и не сводится к
субъективному опыту.

\hypertarget{ux43eux43dux442ux43eux43bux43eux433ux438ux447ux435ux441ux43aux430ux44f-ux43fux443ux441ux442ux43eux442ux430}{%
\subsubsection{2. Онтологическая
Пустота}\label{ux43eux43dux442ux43eux43bux43eux433ux438ux447ux435ux441ux43aux430ux44f-ux43fux443ux441ux442ux43eux442ux430}}

\textbf{Онтологическая Пустота} --- состояние отсутствия определённости,
в котором ещё не выделены никакие различия, формы или структуры, но
потенциально возможны все различения.

Пустота не есть небытие и не является вакуумом в физическом смысле.

\hypertarget{ux43eux43fux440ux435ux434ux435ux43bux451ux43dux43dux43eux441ux442ux44c}{%
\subsubsection{3.
Определённость}\label{ux43eux43fux440ux435ux434ux435ux43bux451ux43dux43dux43eux441ux442ux44c}}

\textbf{Определённость} --- наличие устойчивого различия, позволяющего
выделить форму, состояние или структуру в пределах Сознания.

\hypertarget{ux43dux435ux43eux43fux440ux435ux434ux435ux43bux451ux43dux43dux43eux441ux442ux44c}{%
\subsubsection{4.
Неопределённость}\label{ux43dux435ux43eux43fux440ux435ux434ux435ux43bux451ux43dux43dux43eux441ux442ux44c}}

\textbf{Неопределённость} --- отсутствие фиксированных различий;
фундаментальное состояние Сознания до и вне актов различения.

Неопределённость является онтологической, а не эпистемической.

\hypertarget{ux440ux430ux437ux43bux438ux447ux438ux435}{%
\subsubsection{5.
Различие}\label{ux440ux430ux437ux43bux438ux447ux438ux435}}

\textbf{Различие} --- первичная онтологическая структура, возникающая в
результате самодифференциации Сознания и предшествующая любым объектам,
свойствам и отношениям.

\hypertarget{ux441ux430ux43cux43eux43eux441ux43eux437ux43dux430ux43dux438ux435}{%
\subsubsection{6.
Самоосознание}\label{ux441ux430ux43cux43eux43eux441ux43eux437ux43dux430ux43dux438ux435}}

\textbf{Самоосознание} --- способность Сознания различать само себя без
внешнего наблюдателя или источника.

\hypertarget{ux441ux430ux43cux43eux434ux438ux444ux444ux435ux440ux435ux43dux446ux438ux430ux446ux438ux44f}{%
\subsubsection{7.
Самодифференциация}\label{ux441ux430ux43cux43eux434ux438ux444ux444ux435ux440ux435ux43dux446ux438ux430ux446ux438ux44f}}

\textbf{Самодифференциация} --- онтологический акт возникновения
различий внутри Сознания, в результате которого формируются структуры
знания.

\hypertarget{ux43eux43dux442ux43eux43bux43eux433ux438ux447ux435ux441ux43aux438ux439-ux430ux43aux442}{%
\subsubsection{8. Онтологический
акт}\label{ux43eux43dux442ux43eux43bux43eux433ux438ux447ux435ux441ux43aux438ux439-ux430ux43aux442}}

\textbf{Онтологический акт} --- фундаментальное событие в Сознании, в
котором возникает новое различие, не выводимое из предшествующих
состояний по причинной схеме.

\hypertarget{ux437ux43dux430ux43dux438ux435}{%
\subsubsection{9. Знание}\label{ux437ux43dux430ux43dux438ux435}}

\textbf{Знание} --- форма определённости в Сознании, возникающая как
устойчивая конфигурация различий.

Знание не предполагает субъекта и не тождественно информации.

\hypertarget{ux444ux43eux440ux43cux430}{%
\subsubsection{10. Форма}\label{ux444ux43eux440ux43cux430}}

\textbf{Форма} --- устойчивая структура знания, сохраняющая различие во
времени и допускающая повторяемость.

\hypertarget{ux43dux430ux431ux43bux44eux434ux430ux442ux435ux43bux44c}{%
\subsubsection{11.
Наблюдатель}\label{ux43dux430ux431ux43bux44eux434ux430ux442ux435ux43bux44c}}

\textbf{Наблюдатель} --- локализация знания в Сознании, при которой
осуществляется выбор и фиксация различий.

Наблюдатель не является субъектом в психологическом смысле.

\hypertarget{ux43dux430ux431ux43bux44eux434ux435ux43dux438ux435}{%
\subsubsection{12.
Наблюдение}\label{ux43dux430ux431ux43bux44eux434ux435ux43dux438ux435}}

\textbf{Наблюдение} --- акт выбора и актуализации различия в Сознании,
приводящий к переходу от неопределённости к определённости.

\hypertarget{ux43fux440ux43eux441ux442ux440ux430ux43dux441ux442ux432ux43e}{%
\subsubsection{13.
Пространство}\label{ux43fux440ux43eux441ux442ux440ux430ux43dux441ux442ux432ux43e}}

\textbf{Пространство} --- структурированное множество различий,
допускающее введение меры близости (метрики).

\hypertarget{ux432ux440ux435ux43cux44f}{%
\subsubsection{14. Время}\label{ux432ux440ux435ux43cux44f}}

\textbf{Время} --- процесс изменения различий в Сознании, выражающий
последовательность онтологических актов.

\hypertarget{ux43cux430ux442ux435ux440ux438ux44f}{%
\subsubsection{15. Материя}\label{ux43cux430ux442ux435ux440ux438ux44f}}

\textbf{Материя} --- устойчивая и повторяющаяся конфигурация форм
знания, проявляющаяся как объективная реальность в опыте наблюдателя.

\end{document}
