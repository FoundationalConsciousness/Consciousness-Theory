% Options for packages loaded elsewhere
\PassOptionsToPackage{unicode}{hyperref}
\PassOptionsToPackage{hyphens}{url}
%
\documentclass[
]{article}
\usepackage{amsmath,amssymb}
\usepackage{lmodern}
\usepackage{iftex}
\ifPDFTeX
  \usepackage[T1]{fontenc}
  \usepackage[utf8]{inputenc}
  \usepackage{textcomp} % provide euro and other symbols
\else % if luatex or xetex
  \usepackage{unicode-math}
  \defaultfontfeatures{Scale=MatchLowercase}
  \defaultfontfeatures[\rmfamily]{Ligatures=TeX,Scale=1}
\fi
% Use upquote if available, for straight quotes in verbatim environments
\IfFileExists{upquote.sty}{\usepackage{upquote}}{}
\IfFileExists{microtype.sty}{% use microtype if available
  \usepackage[]{microtype}
  \UseMicrotypeSet[protrusion]{basicmath} % disable protrusion for tt fonts
}{}
\makeatletter
\@ifundefined{KOMAClassName}{% if non-KOMA class
  \IfFileExists{parskip.sty}{%
    \usepackage{parskip}
  }{% else
    \setlength{\parindent}{0pt}
    \setlength{\parskip}{6pt plus 2pt minus 1pt}}
}{% if KOMA class
  \KOMAoptions{parskip=half}}
\makeatother
\usepackage{xcolor}
\setlength{\emergencystretch}{3em} % prevent overfull lines
\providecommand{\tightlist}{%
  \setlength{\itemsep}{0pt}\setlength{\parskip}{0pt}}
\setcounter{secnumdepth}{-\maxdimen} % remove section numbering
\ifLuaTeX
  \usepackage{selnolig}  % disable illegal ligatures
\fi
\IfFileExists{bookmark.sty}{\usepackage{bookmark}}{\usepackage{hyperref}}
\IfFileExists{xurl.sty}{\usepackage{xurl}}{} % add URL line breaks if available
\urlstyle{same} % disable monospaced font for URLs
\hypersetup{
  hidelinks,
  pdfcreator={LaTeX via pandoc}}

\author{}
\date{}

\begin{document}

\hypertarget{ux433ux43bux430ux432ux430-12.-ux441ux432ux43eux431ux43eux434ux430-ux438-ux43dux435ux43eux43fux440ux435ux434ux435ux43bux451ux43dux43dux43eux441ux442ux44c}{%
\section{Глава 12. Свобода и
неопределённость}\label{ux433ux43bux430ux432ux430-12.-ux441ux432ux43eux431ux43eux434ux430-ux438-ux43dux435ux43eux43fux440ux435ux434ux435ux43bux451ux43dux43dux43eux441ux442ux44c}}

\hypertarget{ux43fux440ux43eux431ux43bux435ux43cux430-ux441ux432ux43eux431ux43eux434ux44b-ux432-ux442ux440ux430ux434ux438ux446ux438ux43eux43dux43dux43eux439-ux43eux43dux442ux43eux43bux43eux433ux438ux438}{%
\subsection{12.1. Проблема свободы в традиционной
онтологии}\label{ux43fux440ux43eux431ux43bux435ux43cux430-ux441ux432ux43eux431ux43eux434ux44b-ux432-ux442ux440ux430ux434ux438ux446ux438ux43eux43dux43dux43eux439-ux43eux43dux442ux43eux43bux43eux433ux438ux438}}

Проблема свободы воли традиционно формулируется как противоречие между
детерминизмом и субъективным ощущением выбора.

Если мир полностью детерминирован, то будущее строго следует из
прошлого, и свобода оказывается иллюзией. Если же мир принципиально
случаен, то выбор теряет смысл и ответственность.

Обе позиции возникают из предположения, что фундаментом реальности
является либо материя, либо физические законы.

В рамках Теории Сознания данная дихотомия снимается, поскольку сама
структура реальности иная.

\hypertarget{ux43eux43dux442ux43eux43bux43eux433ux438ux447ux435ux441ux43aux430ux44f-ux43dux435ux43eux43fux440ux435ux434ux435ux43bux451ux43dux43dux43eux441ux442ux44c}{%
\subsection{12.2. Онтологическая
неопределённость}\label{ux43eux43dux442ux43eux43bux43eux433ux438ux447ux435ux441ux43aux430ux44f-ux43dux435ux43eux43fux440ux435ux434ux435ux43bux451ux43dux43dux43eux441ux442ux44c}}

Ключевым понятием данной главы является \textbf{онтологическая
неопределённость}.

Неопределённость в Теории Сознания не является следствием:

\begin{itemize}
\item
  недостатка информации,
\item
  ограниченности наблюдателя,
\item
  скрытых параметров.
\end{itemize}

Она является \textbf{фундаментальным свойством Сознания}.

Пустота, лежащая в основании Сознания, не содержит предопределённых
форм. Формы возникают через самодифференциацию, а не через реализацию
заранее заданного сценария.

Таким образом, будущее не содержится полностью в настоящем.

\hypertarget{ux43dux435ux43eux43fux440ux435ux434ux435ux43bux451ux43dux43dux43eux441ux442ux44c-ux438-ux437ux43dux430ux43dux438ux435}{%
\subsection{12.3. Неопределённость и
знание}\label{ux43dux435ux43eux43fux440ux435ux434ux435ux43bux451ux43dux43dux43eux441ux442ux44c-ux438-ux437ux43dux430ux43dux438ux435}}

Знание, как форма Сознания, всегда является:

\begin{itemize}
\item
  частичным,
\item
  локализованным,
\item
  структурированным.
\end{itemize}

Даже глобальное знание не устраняет неопределённость, поскольку сама
структура различий остаётся открытой к новым конфигурациям.

Неопределённость здесь есть не отсутствие знания, а \textbf{открытость
структуры знания к порождению новых различий}.

\hypertarget{ux432ux440ux435ux43cux44f-ux438-ux43dux435ux437ux430ux43cux43aux43dux443ux442ux43eux441ux442ux44c-ux431ux443ux434ux443ux449ux435ux433ux43e}{%
\subsection{12.4. Время и незамкнутость
будущего}\label{ux432ux440ux435ux43cux44f-ux438-ux43dux435ux437ux430ux43cux43aux43dux443ux442ux43eux441ux442ux44c-ux431ux443ux434ux443ux449ux435ux433ux43e}}

Как было показано ранее, время возникает как процесс изменения различий.

Если различия не заданы полностью, то и временная структура не может
быть полностью предопределена.

Будущее не существует как фиксированное состояние, а формируется в
процессе динамики Сознания.

Следовательно, недоопределённость будущего является онтологической, а не
эпистемологической.

\hypertarget{ux441ux432ux43eux431ux43eux434ux430-ux43aux430ux43a-ux43dux435ux434ux43eux43eux43fux440ux435ux434ux435ux43bux451ux43dux43dux43eux441ux442ux44c}{%
\subsection{12.5. Свобода как
недоопределённость}\label{ux441ux432ux43eux431ux43eux434ux430-ux43aux430ux43a-ux43dux435ux434ux43eux43eux43fux440ux435ux434ux435ux43bux451ux43dux43dux43eux441ux442ux44c}}

В данной онтологии \textbf{свобода} определяется не как произвольность
или отсутствие причин, а как \textbf{недоопределённость будущего
относительно настоящего}.

Свобода не противоречит причинности, поскольку причинность описывает
устойчивые связи, а не полную предзаданность.

Там, где причинные связи не замыкают пространство возможностей,
возникает свобода.

\hypertarget{ux441ux432ux43eux431ux43eux434ux430-ux438-ux43dux430ux431ux43bux44eux434ux430ux442ux435ux43bux44c}{%
\subsection{12.6. Свобода и
наблюдатель}\label{ux441ux432ux43eux431ux43eux434ux430-ux438-ux43dux430ux431ux43bux44eux434ux430ux442ux435ux43bux44c}}

Наблюдатель, как локализация знания, участвует в формировании реальности
через акты выбора и фиксации.

Выбор не создаёт произвольную реальность, но актуализирует одну из
допустимых конфигураций различий.

Таким образом, свобода есть активная роль наблюдателя в процессе
саморазворачивания Сознания.

\hypertarget{ux441ux432ux43eux431ux43eux434ux430-ux438-ux43eux442ux432ux435ux442ux441ux442ux432ux435ux43dux43dux43eux441ux442ux44c}{%
\subsection{12.7. Свобода и
ответственность}\label{ux441ux432ux43eux431ux43eux434ux430-ux438-ux43eux442ux432ux435ux442ux441ux442ux432ux435ux43dux43dux43eux441ux442ux44c}}

Поскольку выбор не является случайным, а укоренён в структуре знания
наблюдателя,\\
свобода не устраняет ответственность.

Ответственность возникает как согласованность выбора с внутренней
структурой различий, а не как внешнее принуждение.

Свобода и ответственность не противопоставлены, а дополняют друг друга.

\hypertarget{ux441ux432ux43eux431ux43eux434ux430-ux438-ux444ux438ux437ux438ux447ux435ux441ux43aux430ux44f-ux43dux435ux43eux43fux440ux435ux434ux435ux43bux451ux43dux43dux43eux441ux442ux44c}{%
\subsection{12.8. Свобода и физическая
неопределённость}\label{ux441ux432ux43eux431ux43eux434ux430-ux438-ux444ux438ux437ux438ux447ux435ux441ux43aux430ux44f-ux43dux435ux43eux43fux440ux435ux434ux435ux43bux451ux43dux43dux43eux441ux442ux44c}}

Квантовая неопределённость является частным проявлением более глубокой
онтологической неопределённости.

Она не создаёт свободу, но указывает на нефундаментальность
классического детерминизма.

Свобода не выводится из квантовой случайности, а укоренена в самой
структуре Сознания.

\hypertarget{ux441ux432ux43eux431ux43eux434ux430-ux43aux430ux43a-ux441ux432ux43eux439ux441ux442ux432ux43e-ux432ux441ux435ux43bux435ux43dux43dux43eux439}{%
\subsection{12.9. Свобода как свойство
Вселенной}\label{ux441ux432ux43eux431ux43eux434ux430-ux43aux430ux43a-ux441ux432ux43eux439ux441ux442ux432ux43e-ux432ux441ux435ux43bux435ux43dux43dux43eux439}}

Свобода не является исключительным свойством человека.

Вся Вселенная, как процесс саморазворачивания Сознания, обладает
степенью свободы, соразмерной её уровню структурированности.

Чем богаче структура различий, тем шире пространство возможных будущих.

\hypertarget{ux432ux44bux432ux43eux434ux44b}{%
\subsection{12.10. Выводы}\label{ux432ux44bux432ux43eux434ux44b}}

В данной главе было показано, что:

\begin{itemize}
\item
  неопределённость является онтологической;
\item
  будущее принципиально незамкнуто;
\item
  свобода есть недоопределённость будущего;
\item
  свобода совместима с причинностью;
\item
  наблюдатель является активным участником реальности.
\end{itemize}

Свобода в Теории Сознанияне является иллюзией и не нарушает законов,
поскольку сами законы являются проявлением устойчивых форм знания, а не
жёстких предписаний.

\end{document}
