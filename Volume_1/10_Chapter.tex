% Options for packages loaded elsewhere
\PassOptionsToPackage{unicode}{hyperref}
\PassOptionsToPackage{hyphens}{url}
%
\documentclass[
]{article}
\usepackage{amsmath,amssymb}
\usepackage{lmodern}
\usepackage{iftex}
\ifPDFTeX
  \usepackage[T1]{fontenc}
  \usepackage[utf8]{inputenc}
  \usepackage{textcomp} % provide euro and other symbols
\else % if luatex or xetex
  \usepackage{unicode-math}
  \defaultfontfeatures{Scale=MatchLowercase}
  \defaultfontfeatures[\rmfamily]{Ligatures=TeX,Scale=1}
\fi
% Use upquote if available, for straight quotes in verbatim environments
\IfFileExists{upquote.sty}{\usepackage{upquote}}{}
\IfFileExists{microtype.sty}{% use microtype if available
  \usepackage[]{microtype}
  \UseMicrotypeSet[protrusion]{basicmath} % disable protrusion for tt fonts
}{}
\makeatletter
\@ifundefined{KOMAClassName}{% if non-KOMA class
  \IfFileExists{parskip.sty}{%
    \usepackage{parskip}
  }{% else
    \setlength{\parindent}{0pt}
    \setlength{\parskip}{6pt plus 2pt minus 1pt}}
}{% if KOMA class
  \KOMAoptions{parskip=half}}
\makeatother
\usepackage{xcolor}
\setlength{\emergencystretch}{3em} % prevent overfull lines
\providecommand{\tightlist}{%
  \setlength{\itemsep}{0pt}\setlength{\parskip}{0pt}}
\setcounter{secnumdepth}{-\maxdimen} % remove section numbering
\ifLuaTeX
  \usepackage{selnolig}  % disable illegal ligatures
\fi
\IfFileExists{bookmark.sty}{\usepackage{bookmark}}{\usepackage{hyperref}}
\IfFileExists{xurl.sty}{\usepackage{xurl}}{} % add URL line breaks if available
\urlstyle{same} % disable monospaced font for URLs
\hypersetup{
  hidelinks,
  pdfcreator={LaTeX via pandoc}}

\author{}
\date{}

\begin{document}

\hypertarget{ux433ux43bux430ux432ux430-10.-ux43dux430ux431ux43bux44eux434ux430ux442ux435ux43bux44c-ux43aux430ux43a-ux43bux43eux43aux430ux43bux438ux437ux430ux446ux438ux44f-ux437ux43dux430ux43dux438ux44f}{%
\section{Глава 10. Наблюдатель как локализация
знания}\label{ux433ux43bux430ux432ux430-10.-ux43dux430ux431ux43bux44eux434ux430ux442ux435ux43bux44c-ux43aux430ux43a-ux43bux43eux43aux430ux43bux438ux437ux430ux446ux438ux44f-ux437ux43dux430ux43dux438ux44f}}

\hypertarget{section}{%
\subsubsection{}\label{section}}

\hypertarget{ux43fux440ux43eux431ux43bux435ux43cux430-ux43dux430ux431ux43bux44eux434ux430ux442ux435ux43bux44f-ux432-ux444ux443ux43dux434ux430ux43cux435ux43dux442ux430ux43bux44cux43dux44bux445-ux442ux435ux43eux440ux438ux44fux445}{%
\subsection{10.1. Проблема наблюдателя в фундаментальных
теориях}\label{ux43fux440ux43eux431ux43bux435ux43cux430-ux43dux430ux431ux43bux44eux434ux430ux442ux435ux43bux44f-ux432-ux444ux443ux43dux434ux430ux43cux435ux43dux442ux430ux43bux44cux43dux44bux445-ux442ux435ux43eux440ux438ux44fux445}}

Понятие наблюдателя занимает центральное место в современной физике и
философии, однако его онтологический статус остаётся неясным.

В классической науке наблюдатель считается внешним по отношению к
наблюдаемой системе.

В квантовой механике, напротив, наблюдатель становится неустранимым
элементом описания, но при этом остаётся концептуально неопределённым.

Возникает парадокс: наблюдатель необходим для описания реальности, но не
имеет ясного онтологического определения.

\hypertarget{ux43dux435ux432ux43eux437ux43cux43eux436ux43dux43eux441ux442ux44c-ux432ux43dux435ux448ux43dux435ux433ux43e-ux43dux430ux431ux43bux44eux434ux430ux442ux435ux43bux44f}{%
\subsection{10.2. Невозможность внешнего
наблюдателя}\label{ux43dux435ux432ux43eux437ux43cux43eux436ux43dux43eux441ux442ux44c-ux432ux43dux435ux448ux43dux435ux433ux43e-ux43dux430ux431ux43bux44eux434ux430ux442ux435ux43bux44f}}

В рамках онтологии Сознания понятие внешнего наблюдателя оказывается
логически противоречивым.

Если Сознание является фундаментальной реальностью, то:

\begin{itemize}
\item
  не существует точки «вне» Сознания;
\item
  любой акт наблюдения происходит внутри структуры различий;
\item
  наблюдатель не может быть трансцендентным по отношению к реальности.
\end{itemize}

Следовательно, наблюдатель должен быть описан как \textbf{внутренняя
структура Сознания}, а не как внешняя сущность.

\hypertarget{ux43eux43dux442ux43eux43bux43eux433ux438ux447ux435ux441ux43aux43eux435-ux43eux43fux440ux435ux434ux435ux43bux435ux43dux438ux435-ux43dux430ux431ux43bux44eux434ux430ux442ux435ux43bux44f}{%
\subsection{10.3. Онтологическое определение
наблюдателя}\label{ux43eux43dux442ux43eux43bux43eux433ux438ux447ux435ux441ux43aux43eux435-ux43eux43fux440ux435ux434ux435ux43bux435ux43dux438ux435-ux43dux430ux431ux43bux44eux434ux430ux442ux435ux43bux44f}}

\textbf{Наблюдатель} определяется как \textbf{локализованная структура
знания}, способная:

\begin{itemize}
\item
  фиксировать различия;
\item
  удерживать устойчивые формы;
\item
  участвовать в процессе самодифференциации.
\end{itemize}

Это определение:

\begin{itemize}
\item
  не антропоцентрично;
\item
  не требует наличия сознания в психологическом смысле;
\item
  применимо к любым уровням реальности.
\end{itemize}

\hypertarget{ux43bux43eux43aux430ux43bux438ux437ux430ux446ux438ux44f-ux43aux430ux43a-ux43eux433ux440ux430ux43dux438ux447ux435ux43dux438ux435-ux440ux430ux437ux43bux438ux447ux438ux439}{%
\subsection{10.4. Локализация как ограничение
различий}\label{ux43bux43eux43aux430ux43bux438ux437ux430ux446ux438ux44f-ux43aux430ux43a-ux43eux433ux440ux430ux43dux438ux447ux435ux43dux438ux435-ux440ux430ux437ux43bux438ux447ux438ux439}}

Локализация наблюдателя означает не пространственное положение, а
\textbf{ограничение множества актуализированных различий}.

Наблюдатель --- это «точка зрения» не в геометрическом, а в
онтологическом смысле.

Он фиксирует лишь подмножество возможных различий, тем самым формируя
наблюдаемую реальность.

\hypertarget{ux430ux43aux442-ux43dux430ux431ux43bux44eux434ux435ux43dux438ux44f}{%
\subsection{10.5. Акт
наблюдения}\label{ux430ux43aux442-ux43dux430ux431ux43bux44eux434ux435ux43dux438ux44f}}

Акт наблюдения представляет собой \textbf{процесс стабилизации
различий}.

До наблюдения формы знания находятся в состоянии неопределённости.
Наблюдение приводит к выбору и фиксации конкретной структуры различий.

Это не добавление информации извне, а внутренний процесс Сознания.

\hypertarget{ux43dux430ux431ux43bux44eux434ux435ux43dux438ux435-ux438-ux438ux437ux43cux435ux440ux435ux43dux438ux435}{%
\subsection{10.6. Наблюдение и
измерение}\label{ux43dux430ux431ux43bux44eux434ux435ux43dux438ux435-ux438-ux438ux437ux43cux435ux440ux435ux43dux438ux435}}

Измерение является частным случаем наблюдения, в котором:

\begin{itemize}
\item
  заранее заданы возможные формы знания;
\item
  результаты представлены в устойчивых символических структурах.
\end{itemize}

Таким образом, измерительный прибор является расширением наблюдателя, а
не независимым объектом.

\hypertarget{ux43cux43dux43eux436ux435ux441ux442ux432ux435ux43dux43dux43eux441ux442ux44c-ux43dux430ux431ux43bux44eux434ux430ux442ux435ux43bux435ux439}{%
\subsection{10.7. Множественность
наблюдателей}\label{ux43cux43dux43eux436ux435ux441ux442ux432ux435ux43dux43dux43eux441ux442ux44c-ux43dux430ux431ux43bux44eux434ux430ux442ux435ux43bux435ux439}}

В рамках данной онтологии возможно существование множества наблюдателей.

Различные локализации знания могут:

\begin{itemize}
\item
  пересекаться;
\item
  согласовываться;
\item
  вступать в противоречие.
\end{itemize}

Объективная реальность возникает как устойчивая структура согласованных
различий между наблюдателями.

\hypertarget{ux43dux430ux431ux43bux44eux434ux430ux442ux435ux43bux44c-ux438-ux43aux432ux430ux43dux442ux43eux432ux430ux44f-ux43dux435ux43eux43fux440ux435ux434ux435ux43bux451ux43dux43dux43eux441ux442ux44c}{%
\subsection{10.8. Наблюдатель и квантовая
неопределённость}\label{ux43dux430ux431ux43bux44eux434ux430ux442ux435ux43bux44c-ux438-ux43aux432ux430ux43dux442ux43eux432ux430ux44f-ux43dux435ux43eux43fux440ux435ux434ux435ux43bux451ux43dux43dux43eux441ux442ux44c}}

Квантовая неопределённость не является фундаментальной случайностью.

Она отражает неполноту локализации различий до акта наблюдения.

Коллапс волновой функции интерпретируется как \textbf{локальная
стабилизация знания}, а не физический скачок.

\hypertarget{ux43fux440ux435ux434ux432ux43eux441ux445ux438ux449ux435ux43dux438ux435-ux444ux43eux440ux43cux430ux43bux438ux437ux430ux446ux438ux438}{%
\subsection{10.9. Предвосхищение
формализации}\label{ux43fux440ux435ux434ux432ux43eux441ux445ux438ux449ux435ux43dux438ux435-ux444ux43eux440ux43cux430ux43bux438ux437ux430ux446ux438ux438}}

В Томе II наблюдатель будет описан формально как:

\begin{itemize}
\item
  подпространство в пространстве состояний;
\item
  оператор проекции;
\item
  механизм редукции неопределённости.
\end{itemize}

Однако формализация не заменяет онтологию, а лишь выражает её на языке
математики.

\hypertarget{ux437ux43dux430ux447ux435ux43dux438ux435-ux433ux43bux430ux432ux44b}{%
\subsection{10.10. Значение
главы}\label{ux437ux43dux430ux447ux435ux43dux438ux435-ux433ux43bux430ux432ux44b}}

В этой главе было показано, что:

\begin{itemize}
\item
  наблюдатель является внутренней структурой Сознания;
\item
  наблюдение --- это процесс локализации и стабилизации знания;
\item
  объективность возникает из согласования наблюдений;
\item
  квантовые парадоксы получают онтологическое объяснение.
\end{itemize}

\end{document}
