\chapter{Метрики различия и геометрия пространства состояний}

\section{Необходимость метрики различия}

В предыдущей главе различие было формализовано
как информационная и вероятностная величина.
Однако для описания структуры пространства состояний
необходимо ввести количественную меру \emph{близости} и \emph{удалённости}
между формами знания.

Метрика различия позволяет:
\begin{itemize}
    \item сравнивать состояния между собой;
    \item вводить геометрические структуры;
    \item формализовать динамику как движение в пространстве состояний.
\end{itemize}

\section{Требования к метрикам}

Пусть $p = \{p_i\}$ и $q = \{q_i\}$ —
два вероятностных распределения состояний.

Функция различия $D(p,q)$ должна:
\begin{itemize}
    \item быть неотрицательной: $D(p,q) \ge 0$;
    \item обращаться в ноль при совпадении распределений;
    \item возрастать при увеличении различия.
\end{itemize}

Важно отметить:
не все меры различия являются метриками в строгом математическом смысле,
однако они сохраняют онтологическую интерпретируемость.

\section{Дивергенция Кульбака–Лейблера}

Одной из фундаментальных мер различия является
дивергенция Кульбака–Лейблера:
\[
D_{\mathrm{KL}}(p \Vert q) = \sum_i p_i \log \frac{p_i}{q_i}.
\]

Эта величина измеряет,
насколько распределение $q$
отклоняется от распределения $p$.

В онтологическом смысле
$D_{\mathrm{KL}}$ характеризует
потерю или приобретение знания
при переходе между формами.

\section{Асимметрия и направленность}

Дивергенция Кульбака–Лейблера асимметрична:
\[
D_{\mathrm{KL}}(p \Vert q) \neq D_{\mathrm{KL}}(q \Vert p).
\]

Эта асимметрия имеет онтологическое значение:
она отражает направленность процессов изменения различий
и связана со стрелой времени.

Таким образом,
временная направленность
получает геометрическое выражение.

\section{Симметризованные меры различия}

Для построения геометрии
часто используются симметризованные меры.

Одной из них является
дивергенция Йенсена–Шеннона:
\[
D_{\mathrm{JS}}(p,q)
= \frac{1}{2} D_{\mathrm{KL}}(p \Vert m)
+ \frac{1}{2} D_{\mathrm{KL}}(q \Vert m),
\quad
m = \frac{1}{2}(p+q).
\]

$D_{\mathrm{JS}}$ обладает ограниченностью
и симметрией,
что делает её удобной
для геометрической интерпретации.

\section{Расстояние Хеллингера}

Ещё одной важной метрикой является расстояние Хеллингера:
\[
H(p,q) = \frac{1}{\sqrt{2}}
\sqrt{
\sum_i \left( \sqrt{p_i} - \sqrt{q_i} \right)^2
}.
\]

Это расстояние:
\begin{itemize}
    \item является истинной метрикой;
    \item тесно связано с геометрией гильбертова пространства;
    \item естественно интерпретируется как различие форм знания.
\end{itemize}

\section{Геометрия пространства состояний}

Введение метрик позволяет рассматривать
пространство состояний как геометрический объект.

Состояния интерпретируются как точки,
а динамика — как траектории,
минимизирующие или экстремизирующие функционалы различия.

Таким образом,
эволюция Сознания
может быть представлена
как геометрический процесс.

\section{Градиенты различия и динамика}

Изменение состояний может быть направлено
в сторону уменьшения или увеличения различия.

Градиент энтропии или дивергенции
задаёт направление естественной динамики.

Это создаёт основу
для формализации «гравитации смысла»,
где системы стремятся к снижению
информационного напряжения.

\section{Связь с физическими пространствами}

Физическое пространство и время
являются проекциями
геометрии пространства состояний
на устойчивые подпространства.

Метрика пространства–времени
возникает как эффективное описание
информационной геометрии Сознания.

\section{Выводы}

В данной главе было показано, что:
\begin{itemize}
    \item различие может быть выражено через метрики;
    \item дивергенции задают геометрию пространства состояний;
    \item направленность времени связана с асимметрией метрик;
    \item динамика может быть интерпретирована геометрически.
\end{itemize}

Это подготавливает введение
понятий наблюдателя, проекций и коллапса
в строгом математическом виде.
