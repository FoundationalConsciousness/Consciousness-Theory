\chapter{Динамика состояний и рождение времени}

\section{Проблема времени в формализме}

В классической и квантовой физике время, как правило, вводится как внешний параметр,
по отношению к которому описывается эволюция системы.
Такой подход удобен вычислительно,
но он не отвечает на вопрос об онтологическом происхождении времени.

В рамках Теории Сознания время не может быть введено априори,
поскольку в Томе~I оно было определено
как процесс изменения различий,
а не как независимая сущность.

Цель данной главы — показать,
как формальное время возникает из динамики состояний в пространстве Сознания.

\section{Динамика как отображение состояний}

Пусть пространство состояний Сознания задано гильбертовым пространством $\mathcal{H}$.
Динамика определяется как правило перехода между состояниями:
\[
|\psi\rangle \longrightarrow |\psi'\rangle.
\]

Формально динамика может быть задана семейством операторов:
\[
\mathcal{U} = \{ U(\lambda) \},
\]
где параметр $\lambda$ первоначально не интерпретируется как время.

Преобразование состояния имеет вид:
\[
|\psi(\lambda)\rangle = U(\lambda) |\psi(0)\rangle.
\]

\section{Унитарность и сохранение структуры}

Для сохранения нормировки и вероятностной интерпретации
операторы динамики предполагаются унитарными:
\[
U^\dagger(\lambda) U(\lambda) = I.
\]

Унитарность отражает онтологический принцип:
изменения различий не уничтожают знание,
а лишь перераспределяют его структуру.

\section{Появление параметра времени}

Если параметр $\lambda$ допускает упорядочивание
и последовательную композицию преобразований:
\[
U(\lambda_1 + \lambda_2) = U(\lambda_1) U(\lambda_2),
\]
то он может быть интерпретирован как параметр эволюции.

Именно на этом этапе параметр $\lambda$ приобретает смысл времени:
\[
\lambda \equiv t.
\]

Таким образом, время возникает как параметр,
упорядочивающий динамику состояний,
а не как фундаментальная координата.

\section{Генератор динамики}

Для непрерывной динамики оператор эволюции может быть представлен в виде:
\[
U(t) = e^{- i \hat{H} t}.
\]

Оператор $\hat{H}$ является генератором изменений состояний
и в дальнейшем будет интерпретирован как оператор энергии.

Важно подчеркнуть:
$\hat{H}$ не вводит время,
а лишь задаёт структуру изменений,
которые могут быть параметризованы временем.

\section{Уравнение эволюции}

Из определения оператора эволюции следует уравнение:
\[
i \frac{d}{dt} |\psi(t)\rangle = \hat{H} |\psi(t)\rangle.
\]

Это уравнение является формальным выражением
онтологического принципа изменения различий.

Оно не описывает движение объектов,
а динамику форм знания в пространстве состояний.

\section{Время как внутренняя величина}

Поскольку время возникает из динамики,
оно является внутренней величиной системы.

Различные подсистемы Сознания могут обладать
различными параметрами эволюции,
что приводит к понятию локального времени.

Глобальное время является приближением,
справедливым для согласованных и устойчивых структур.

\section{Необратимость и симметрия}

Формальное уравнение эволюции симметрично по времени.
Однако необратимость возникает при рассмотрении:
\begin{itemize}
    \item устойчивых состояний;
    \item процессов локализации;
    \item редукции неопределённости.
\end{itemize}

Таким образом, стрела времени
является следствием онтологической структуры,
а не нарушения фундаментальных уравнений.

\section{Связь с онтологией Тома I}

Полученный формализм напрямую соответствует выводам Тома~I:
\begin{itemize}
    \item время есть процесс изменения различий;
    \item динамика первична по отношению к времени;
    \item устойчивость форм порождает направленность времени.
\end{itemize}

Математическое время является параметризацией
онтологического процесса.

\section{Выводы}

В данной главе было показано, что:
\begin{itemize}
    \item динамика состояний может быть введена без априорного времени;
    \item время возникает как параметр эволюции состояний;
    \item генератор динамики задаёт структуру изменений;
    \item формализм согласован с онтологией Сознания.
\end{itemize}

Это создаёт основу для введения энергии,
энтропии и информационных метрик,
которые будут рассмотрены в следующих главах.
