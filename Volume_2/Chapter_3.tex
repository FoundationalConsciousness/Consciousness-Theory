\chapter{Энергия, генераторы и законы сохранения}

\section{Энергия как производная величина}

В классической физике энергия часто трактуется как фундаментальная величина,
определяющая динамику системы.
Однако в рамках Теории Сознания энергия не является первоосновой,
а возникает как характеристика динамики состояний.

Цель данной главы — формально вывести понятие энергии
как генератора изменений в пространстве состояний
и показать происхождение законов сохранения.

\section{Генератор динамики}

В Главе~2 было показано,
что непрерывная динамика состояний может быть представлена оператором:
\[
U(t) = e^{- i \hat{H} t}.
\]

Оператор $\hat{H}$ определяет структуру изменений состояний
и является генератором динамики.

Важно подчеркнуть:
$\hat{H}$ не описывает «энергию объекта»,
а задаёт допустимые траектории эволюции форм знания.

\section{Собственные состояния и спектр}

Рассмотрим собственные состояния оператора $\hat{H}$:
\[
\hat{H} |E_n\rangle = E_n |E_n\rangle.
\]

Собственные значения $E_n$ интерпретируются
как устойчивые моды динамики.

Состояние, разложенное по собственным векторам,
\[
|\psi\rangle = \sum_n c_n |E_n\rangle,
\]
представляет собой суперпозицию динамических режимов.

\section{Энергия и устойчивость}

Устойчивые формы знания соответствуют состояниям,
для которых динамика минимально изменяет структуру различий.

Низкие значения энергии связаны с высокой устойчивостью,
в то время как высокоэнергетические состояния
соответствуют интенсивным изменениям различий.

Таким образом, энергия количественно выражает
интенсивность динамики.

\section{Среднее значение энергии}

Среднее значение энергии в состоянии $|\psi\rangle$
определяется как:
\[
\langle E \rangle = \langle \psi | \hat{H} | \psi \rangle.
\]

Это выражение описывает не энергию как субстанцию,
а среднюю скорость изменений различий
в данной локализации знания.

\section{Законы сохранения}

Если оператор $\hat{H}$ не зависит от параметра времени,
то среднее значение энергии сохраняется:
\[
\frac{d}{dt} \langle E \rangle = 0.
\]

Закон сохранения энергии возникает как следствие
структурной неизменности генератора динамики.

Это отражает онтологический принцип:
устойчивые структуры различий
порождают устойчивые количественные характеристики.

\section{Симметрии и сохранение}

Пусть оператор $\hat{G}$ генерирует непрерывную симметрию:
\[
U(\alpha) = e^{- i \hat{G} \alpha}.
\]

Если:
\[
[\hat{H}, \hat{G}] = 0,
\]
то соответствующая величина сохраняется.

Таким образом, законы сохранения являются следствием симметрий
в структуре различий.

\section{Причинность и генераторы}

Генератор $\hat{H}$ определяет допустимые переходы
между состояниями.

Причинность возникает как устойчивость этих переходов
при повторяемости динамики.

Формально причинность не требует жёсткой детерминации,
а выражается в высокой вероятности определённых траекторий.

\section{Связь с онтологией Тома I}

Полученные результаты напрямую соответствуют Главе~11 Тома~I:
\begin{itemize}
    \item энергия как динамика изменений;
    \item причинность как статистическая устойчивость;
    \item сохранение как следствие устойчивости форм.
\end{itemize}

Формализм не вводит новых сущностей,
а уточняет онтологически выведенные понятия.

\section{Выводы}

В данной главе было показано, что:
\begin{itemize}
    \item энергия является генератором динамики состояний;
    \item собственные значения энергии связаны с устойчивостью;
    \item законы сохранения следуют из симметрий;
    \item причинность имеет вероятностную природу.
\end{itemize}

Это подготавливает почву
для введения энтропии, информации и метрик различия.
