\chapter{Сознание как пространство состояний}

\section{Задача математической формализации}

Целью данного тома является не редукция Сознания к математическим объектам,
а формальное описание тех структур, которые были онтологически выведены в Томе~I.

Математика здесь выступает не как первооснова,
а как язык описания устойчивых структур различий в Сознании.

Основной задачей настоящей главы является введение пространства состояний,
в котором Сознание может быть представлено формально.

\section{Сознание и множество возможных состояний}

Пусть $\mathcal{C}$ обозначает Сознание как фундаментальную реальность.
Сознание не является состоянием само по себе,
однако оно допускает множество возможных актуализаций.

Определим множество возможных состояний:
\[
\mathcal{S} = \{ s_i \mid s_i \text{ — возможная конфигурация знания} \}.
\]

Каждое состояние $s \in \mathcal{S}$ соответствует определённой структуре различий,
актуализированной в Сознании.

\section{Состояние как форма знания}

Состояние $s$ интерпретируется как форма знания,
то есть как устойчивый паттерн различий,
доступный локализации наблюдателя.

Важно подчеркнуть, что:
\begin{itemize}
    \item состояния не являются объектами;
    \item они не существуют независимо от Сознания;
    \item они являются внутренними конфигурациями различий.
\end{itemize}

Таким образом, пространство состояний не является «вместилищем»,
а представляет собой формальное описание возможных форм знания.

\section{Линейная структура пространства состояний}

Для описания суперпозиции форм знания
предполагается, что пространство состояний обладает линейной структурой.

Рассмотрим линейное пространство $\mathcal{H}$ над полем $\mathbb{C}$,
элементы которого соответствуют состояниям Сознания:
\[
|\psi\rangle \in \mathcal{H}.
\]

Линейная комбинация состояний
\[
|\psi\rangle = \sum_i c_i |s_i\rangle
\]
интерпретируется как суперпозиция форм знания,
а не как физическое наложение объектов.

\section{Нормировка и вероятность}

Для введения вероятностной интерпретации
в пространстве $\mathcal{H}$ вводится скалярное произведение
\[
\langle \phi | \psi \rangle.
\]

Нормировка состояния:
\[
\langle \psi | \psi \rangle = 1
\]
отражает полноту актуализации знания в рамках данной локализации наблюдателя.

Вероятность актуализации состояния $|s_i\rangle$
определяется как:
\[
P(s_i) = |\langle s_i | \psi \rangle|^2.
\]

\section{Суперпозиция как онтологическое свойство}

Суперпозиция не является следствием неполноты знания наблюдателя.

Она отражает реальное онтологическое состояние до локализации:
несколько форм знания сосуществуют как потенциальные различия.

Таким образом, принцип суперпозиции является
математическим выражением онтологической неопределённости.

\section{Пространство состояний и различие}

Различие между состояниями $|s_i\rangle$ и $|s_j\rangle$
может быть формализовано через расстояние в пространстве состояний.

Это открывает путь к введению:
\begin{itemize}
    \item метрик;
    \item информационных дивергенций;
    \item геометрических интерпретаций.
\end{itemize}

Эти инструменты будут подробно рассмотрены в последующих главах.

\section{Онтологический статус гильбертова пространства}

Важно подчеркнуть:
гильбертово пространство не является «реальным пространством».

Оно представляет собой:
\begin{itemize}
    \item формальную модель структуры различий;
    \item язык описания форм знания;
    \item средство вычисления вероятностей и динамики.
\end{itemize}

Онтология остаётся первичной по отношению к формализму.

\section{Переход к динамике}

В данной главе пространство состояний было введено как статическая структура.

В следующей главе будет показано,
как изменение различий приводит к динамике состояний
и к появлению формального времени.

\section{Выводы}

В этой главе было показано, что:
\begin{itemize}
    \item Сознание допускает формальное описание через пространство состояний;
    \item состояния интерпретируются как формы знания;
    \item суперпозиция отражает онтологическую неопределённость;
    \item вероятностная структура является фундаментальной.
\end{itemize}

Это создаёт основу для строгой математической динамики,
которая будет рассмотрена далее.
