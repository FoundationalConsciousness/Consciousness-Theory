\chapter{Корреляции, нелокальность и структура связей}

\section{Проблема нелокальности}

Квантовая нелокальность традиционно рассматривается
как одно из самых парадоксальных свойств физической реальности.
Корреляции между удалёнными системами
кажутся нарушающими принципы локальности и причинности.

В рамках Теории Сознания
данный парадокс возникает лишь при предположении,
что физическое пространство является фундаментальным.

Цель данной главы —
показать, что нелокальность является
естественным следствием структуры пространства состояний,
а не физическим «действием на расстоянии».

\section{Совместные состояния и тензорное произведение}

Рассмотрим две подсистемы
с пространствами состояний $\mathcal{H}_A$ и $\mathcal{H}_B$.

Совместное пространство состояний задаётся
тензорным произведением:
\[
\mathcal{H}_{AB} = \mathcal{H}_A \otimes \mathcal{H}_B.
\]

Состояние системы может быть либо факторизуемым:
\[
|\psi\rangle = |\psi_A\rangle \otimes |\psi_B\rangle,
\]
либо нефакторизуемым,
что соответствует коррелированным формам знания.

\section{Запутанность как структура знания}

Состояние $|\psi\rangle \in \mathcal{H}_{AB}$
называется запутанным,
если оно не может быть представлено
в виде произведения состояний подсистем.

Запутанность отражает
не физическую связь между объектами,
а целостную структуру знания,
которая не разлагается
на независимые компоненты.

\section{Корреляции наблюдений}

Пусть наблюдатели $A$ и $B$
осуществляют проекционные измерения
$\hat{P}_i^A$ и $\hat{P}_j^B$.

Совместная вероятность результатов равна:
\[
P(i,j) =
\langle \psi |
\hat{P}_i^A \otimes \hat{P}_j^B
| \psi \rangle.
\]

Корреляции возникают
как следствие структуры совместного состояния,
а не в результате передачи сигналов.

\section{Нелокальность без передачи информации}

Важно подчеркнуть:
изменение локализации знания у наблюдателя $A$
не передаёт управляемую информацию наблюдателю $B$.

Редукция состояния является
обновлением описания формы знания,
а не физическим воздействием.

Это устраняет противоречие
между нелокальностью и причинностью.

\section{Геометрическая интерпретация корреляций}

В пространстве состояний
запутанные состояния располагаются
близко друг к другу
в смысле информационных метрик,
даже если соответствующие
физические проекции удалены.

Таким образом,
нелокальность есть проявление
близости в пространстве различий,
а не в физическом пространстве.

\section{Энтропия и корреляции}

Для подсистемы $A$
вводится редуцированная матрица плотности:
\[
\rho_A = \mathrm{Tr}_B(\rho_{AB}).
\]

Энтропия фон Неймана:
\[
S(\rho_A) = - \mathrm{Tr}(\rho_A \log \rho_A)
\]
характеризует степень запутанности
и утрату локального знания.

Рост энтропии подсистемы
не означает рост глобальной неопределённости,
а отражает перераспределение различий.

\section{Причинность и корреляции}

Корреляции могут быть сильными,
но они не нарушают причинность,
поскольку не допускают
контролируемой передачи информации.

Причинность сохраняется
на уровне динамики и наблюдения,
в то время как корреляции
относятся к структуре знания.

\section{Связь с онтологией Тома I}

Результаты данной главы
непосредственно следуют
из онтологических принципов:
\begin{itemize}
    \item пространство — структура различий;
    \item наблюдатель — локализация знания;
    \item нелокальность — близость форм знания.
\end{itemize}

Физическое расстояние
не является фундаментальной мерой связи.

\section{Выводы}

В данной главе было показано, что:
\begin{itemize}
    \item корреляции возникают из целостных состояний;
    \item запутанность имеет информационную природу;
    \item нелокальность не нарушает причинность;
    \item структура связей определяется геометрией пространства состояний.
\end{itemize}

Это завершает формализацию квантовых корреляций
и подготавливает переход
к рассмотрению классического предела
и возникновения макроскопической реальности.
